%-------------------------------------------------------------------------------
% cfg
%-------------------------------------------------------------------------------
%
% \file        cfg.tex
% \library     Documents
% \author      Chris Ahlstrom
% \date        2024-04-16
% \update      2024-07-22
% \version     $Revision$
% \license     $XPC_GPL_LICENSE$
%
%     Provides a description of the entities in the cfg66 library, in the
%     "cfg" namespace.
%
%-------------------------------------------------------------------------------

\section{Cfg Namespace}
\label{sec:cfg_namespace}

   This section provides a useful walkthrough of the \texttt{cfg} namespace of
   the \textsl{cfg66} library.

   Here are the classes (or modules) in this namespace:

   \begin{itemize}
      \item \texttt{cfg::appinfo}
      \item \texttt{cfg::basesettings}
      \item \texttt{cfg::comments}
      \item \texttt{cfg::configfile}
      \item \texttt{cfg::history}
      \item \texttt{cfg::inifile}
      \item \texttt{cfg::inimanager}
      \item \texttt{cfg::inisection}
      \item \texttt{cfg::inisections}
      \item \texttt{cfg::memento}
      \item \texttt{cfg::options}
      \item \texttt{cfg::palette}
      \item \texttt{cfg::recent}
   \end{itemize}

\subsection{cfg::appinfo}
\label{subsec:cfg_namespace_appinfo}

   The \texttt{cfg::appinfo} structure encapsulates basic information about
   an application:

   \begin{itemize}
      \item \textsl{cfg::appkind}.
         Indicates if the application is a headless application (such as a
         daemon), a command-line, \textsl{ncurses}, GUI, or test application.
         In some cases it can be useful to know how the application is
         running.
      \item \textsl{App name}.
         Provides the short name of the application, which can be shown
         in warning messages or be used to name the application as a node, for
         example, in \textsl{JACK}.
      \item \textsl{App version}.
         Provides a version number, or optionally, a variant on the
         \textsl{Name} plus the version number.
      \item \textsl{Main config section name}.
         Holds the name of main configuration section (usually present
         only in the "session" or "rc" file).
         The default value is "[Cfg66]".
      \item \textsl{App home configuration directory}.
         Provides the name of the configuration directory for the application,
         such as \texttt{/home/user/.config/appname}.
         Modified by the \texttt{--home} option.
      \item \textsl{Home configuration file}.
         Provides the name of the main configuration file, such as
         \texttt{appname.rc}.
         Modified by the \texttt{--config} option.
      \item \textsl{Client name}.
         Either the short application name or a variant of it, useful in session
         managers, for example.
         An example under \textsl{NSM} would be \texttt{seq66v2.nUKIE}.
      \item \textsl{App tag}.
         The short application name with the version number tacked on.
      \item \textsl{Arg0}.
         Holds the complete path to the executable as determined at run-time.
      \item \textsl{Package name}.
         Provides the name of the package, which could be the same as the
         application or library name.
      \item \textsl{Session Tag}.
         Provides the session name for the application; it can be the
         application name, a modification of the application name, or
         something completely different, such as a session name given by
         a user.
         Useful in long error/warning/info messages.
      \item \textsl{App icon}.
         Provides the base name of the application icon, if not empty.
      \item \textsl{Version text}.
         Reconstructed version information for the application.
      \item \textsl{API engine}.
         Some applications might use various libraries.
         For example, for a MIDI application it might be one of these
         MIDI libraries:
         \texttt{rtmidi},
         \texttt{rtl66},
         or \texttt{portmidi}.
      \item \textsl{API version}.
         Indicates which version of an API is in force.
         In some cases this can be detected at run-time.
      \item \textsl{GUI version}.
         Indicates the GUI or "curses" version, such as \textsl{Qt 6.1}
         or \textsl{Gtkmm 3.0}.
      \item \textsl{Short client name}.
         Similar to \textsl{client name}, but never has anything appended
         to it.
      \item \textsl{Client name tag}.
         Provides the name to show on the console in error/warning/info messages,
         the short client name surrounded by brackets, such as
         \texttt{[seq66v2]}.
   \end{itemize}

   All of these items can be set at once using the \texttt{appinfo}
   constructor that has a large number of parameters.
   In addition, there are a many "free" functions in the \texttt{cfg}
   namespace for setting and getting these values.
   See \texttt{appinfo.hpp} for a summary.

\subsection{cfg::basesettings}
\label{subsec:cfg_namespace_basesettings}

   Provides common settings useful in any application.

   \begin{itemize}
      \item \textsl{File name}.
         Holds the (optional) name of the file that holds the settings data.
      \item \textsl{Modified}.
         Indicates if the settings have been modified.
      \item \textsl{Config format}.
         A free-form string indicating the format of the data, such as
         \texttt{INI}, \texttt{XML}, or \texttt{JSON}.
         For "66" libraries and applications, the \texttt{INI}
         format is sufficient.
      \item \textsl{Config type}.
         A short string that indicates something about the
         format or content of the file. For example, in Seq66,
         common values were 'rc' and 'usr', with these
         values also representing the file extension.
      \item \textsl{Ordinal version}.
         A simple integer that is incremented each time a change is
         made in the configuration format or values-present
         that isn't detectable during parsing.
      \item \textsl{Comments}.
         An object holding the main comments that describe something
         about the settings file.
      \item \textsl{Is error}.
         Indicated if there was a error during parsing.
      \item \textsl{Error message}.
         If an error occurred, the error message is stored for display.
   \end{itemize}

\subsection{cfg::comments}
\label{subsec:cfg_namespace_comments}

   Holds a string describing something general about the configuration.
   Provides some setter and getter functions.

\subsection{cfg::configfile}
\label{subsec:cfg_namespace_configfile}

   Provides some items common to configuration files, including
   an extensive set of functions to parse sections and configuration
   variables in an
   INI format in a line-by-line fashion.

   These are the main externally-accessible values:

   \begin{itemize}
      \item \textsl{File extension}.
         Common values are 'rc' and 'session'.
      \item \textsl{File name}.
         Provides the file name, normally as a full-path file-specification.
      \item \textsl{File version}.
         Provides the current version of the derived configuration file format.
         Set in the constructor of the configfile-derived object,
         and incremented in that object whenever a new way of reading,
         writing, or formatting the configuration file is created.
   \end{itemize}

   Also too many to list, but it includes functions such as
   \texttt{get\_integer()} and \texttt{write\_integer()}.

   These work by having the whole file read into an
   \texttt{std::ifstream}, and then searching the string over and over
   to read all the variables.
   Sounds inefficient, but in practise it is very fast.

   Finally, there are free functions to delete a configuration file and
   to make a copy of a configuration file.

\subsection{cfg::history}
\label{subsec:cfg_namespace_history}

   One of the things not handled so well in \textsl{Seq66} is the
   undo/redo functionality.
   The \texttt{history} template class implements undo/redo using the 
   \texttt{memento} class described below. It follows
   the \textsl{Design Patterns} book (\cite{dpatterns}) starting on page 62.
   Also informative is \cite{deque}.
   Also see the \texttt{cfg::memento} class below.

   The heart of the \texttt{history} template is the history list:

   \begin{verbatim}
      std::deque<memento<TYPE>> m_history_list;
   \end{verbatim}

   Member functions are provided to see if history entries are
   undoable/redoable, undo and redo them, check the maximum size of the list,
   The \texttt{history.cpp} module provide a small test of history for
   the \texttt{cfg::options} class.

\subsection{cfg::inifile}
\label{subsec:cfg_namespace_inifile}

   \texttt{cfg::inifile} is derived from the \texttt{cfg::configfile} class.
   It contains a reference to a class that manages multiple
   INI sections: \texttt{cfg::inisections}.
   It override the
   \texttt{cfg::configfile::parse()} and
   \texttt{cfg::configfile::write()} functions.
   It provides two protected functions,
   \texttt{parse\_section()} and
   \texttt{write\_section()}.

   WALK THROUGH the ini\_test application.

\subsection{cfg::inimanager}
\label{subsec:cfg_namespace_inimanager}
         
   \texttt{cfg::inimanager} defines two types:

   \begin{itemize}
      \item \texttt{optionref}.
         \texttt{std::reference\_wrapper<options::spec>}.
      \item \texttt{optionmap}.
         \texttt{std::map<std::string, optionref>}.
   \end{itemize}

   Also defined is a function to add an named \texttt{cfg::option}
   to the option map.

\subsection{cfg::inisection}
\label{subsec:cfg_namespace_inisection}

   \texttt{cfg::inisection}  contains a \texttt{specification} structure that
   provides the name of a section, it's description, and
   a container of options.
   It is represent in a configuration file by a section or tag name
   enclosed in brackets:  \texttt{[output-ports]} as an example.
   Each section can also define a file extension
   (e.g. \texttt{.rc}) to indicate its locus.

\subsection{cfg::inisections}
\label{subsec:cfg_namespace_inisections}

   The \texttt{inisections.hpp} file has been split,
   and no long declares four classes.

   \texttt{cfg::inisections} defines four types:

   \begin{itemize}
      \item \texttt{specref}.
          \texttt{std::reference\_wrapper<inisection::specification>}.
      \item \texttt{specrefs}.
          \texttt{std::vector<specref>}.
      \item \texttt{sectionlist}.
         \texttt{std::vector<inisection>}.
      \item \texttt{specification}.
         This structure holds the configuration file's extension,
         directory, base name, description, and a vector of
         \texttt{specrefs}.
   \end{itemize}

   \texttt{cfg::inisections} holds a list of \texttt{inisection} objects.
   It represents all of the files, and all of
   the options that are held by the files.
   The options are in a single list,
   and the INI items look them up by name.

   There is a lot to this module.
   For now, see the comment in the \texttt{.hpp} and \texttt{.cpp} files.

\subsection{cfg::memento}
\label{subsec:cfg_namespace_memento}

   The \texttt{cfg::memento} template class is a small object that holds one
   copy of a state object.
   It provides these functions:

   \begin{itemize}
      \item \texttt{memento (const TYPE \& s)}.
      \item \texttt{bool set\_state (const TYPE \& s)}.
      \item \texttt{const TYPE \& get\_state () const}.
   \end{itemize}

   The \texttt{cfg::history} class stores a "history list":
   \texttt{std::deque<memento<TYPE>>}.

\subsection{cfg::options}
\label{subsec:cfg_namespace_options}

   The \texttt{cfg::options} holds a number of items needed for the
   specification, reading, and writing of options.
   The options can be read from configuration file or from the command-line.
   These items are nested in the class:

   \begin{itemize}
      \item \textsl{Static data}.
         Flags and numbers are provided to indicate if the option is enabled
         and how it is to be output in a nice format into an INI file.
      \item \textsl{kind}.
         Indicates if the option is boolean, numerical, a filename, list,
         string and some others. This enumeration makes it easier to
         process the option.
      \item \textsl{spec}.
         This nested structure contains these values. For brevity, the
         \texttt{option\_} portion of the name and the type are not shown:
         \begin{itemize}
            \item \texttt{code}.
               Optional single-character name.
            \item \texttt{kind}.
               Is it boolean, integer, string...?
            \item \texttt{cli\_enabled}.
               Normally true; false disables.
            \item \texttt{default}.
               Either a value or "true"/"false".
            \item \texttt{value}.
               The actual value as parsed.
            \item \texttt{read\_from\_cli}.
               Option already set from CLI.
            \item \texttt{modified}.
               Option changed since read/save.
            \item \texttt{desc}.
               A one-line description of option.
            \item \texttt{built\_in}.
               This option is present in all apps.
         \end{itemize}
      \item \textsl{option}.
         The \texttt{cfg::options::option} is a simple pair of the
         name of the option and the \texttt{spec} that describes it.
      \item \textsl{container}.
         The \texttt{cfg::options::container} is a map (dictionary) of
         option \texttt{spec}s keyed bythe name of the option.
   \end{itemize}

   Also specified are the name of the source file and the name of the
   source section in that file.

   Included are quite a number of functions for looking up option values
   and option characteristics.
   Also include are free functions to make options.

   The \texttt{options.cpp} module not only contains many comments explaining
   the module, but also a statically-initialized list of
   default options that any application can use.
   It is also a great example of how to creat a list of options.

\subsection{cfg::palette}
\label{subsec:cfg_namespace_}

   The \texttt{cfg::palette} template class can be used to define
   a palette of color code pair with a platform-specific color class
   such as \texttt{QColor} from \textsl{Qt}.
   This is a feature copped from \textsl{Seq66}.

\subsection{cfg::recent}
\label{subsec:cfg_namespace_recent}

   \texttt{recent} provides for the managment of a list of recently-used
   items.

   TODO.

%-------------------------------------------------------------------------------
% vim: ts=3 sw=3 et ft=tex
%-------------------------------------------------------------------------------

%-------------------------------------------------------------------------------
% tests
%-------------------------------------------------------------------------------
%
% \file        tests.tex
% \library     Documents
% \author      Chris Ahlstrom
% \date        2024-04-16
% \update      2024-04-16
% \version     $Revision$
% \license     $XPC_GPL_LICENSE$
%
%     Provides a pointed description of some tests, which also helps understand
%     usage of the cfg66 library.
%
%-------------------------------------------------------------------------------

\section{Cfg66 Tests}
\label{sec:cfg66_tests}

   This section provides a useful walkthrough of the testing
   of the \textsl{cfg66} library.
   They illustrate the various ways in which the \textsl{Cfg66} library
   can be used by a developer.

   The tests so far are these executables:

    \begin{itemize}
        \item \texttt{cliparser\_test\_c}
        \item \texttt{cliparser\_test}
        \item \texttt{history\_test}
        \item \texttt{ini\_test}
        \item \texttt{manager\_test}
        \item \texttt{options\_test}
    \end{itemize}

    These tests are supported by data structures define in the following
    header files:

    \begin{itemize}
        \item texttt{ctrl\_spec.hpp}
        \item texttt{drums\_spec.hpp}
        \item texttt{mutes\_spec.hpp}
        \item texttt{palette\_spec.hpp}
        \item texttt{playlist\_spec.hpp}
        \item texttt{rc\_spec.hpp}
        \item texttt{session\_spec.hpp}
        \item texttt{test\_spec.hpp}
        \item texttt{usr\_spec.hpp}
    \end{itemize}

    These header files will be discussed as needed in the following sections.

\subsection{Cfg66 cliparser\_test\_c Test}
\label{subsec:cfg66_tests_cliparser_test_c}

    This test of the \texttt{C} command-line interface uses the free
    functions in \texttt{cliparser\_c.h}. The test module itself
    sets up a small set of test options:

    \begin{verbatim}
    static options_spec s_test_options [] =
    {
        //   Name          Code  Kind     Enabled   Default     Value      Dirty
        {
            "alertable",    'a', "boolean", true,   "false",    "false",   false,
            "If specified, the application is alertable."
        }, . . .
    };
    \end{verbatim}

    The test makes changes to the options and verifies that they
    took hold.
    The test command is simple:

    \begin{verbatim}
        $ ./build/test/cliparser_test_c
    \end{verbatim}

    It shows the changes and a result statement.

\subsection{Cfg66 cliparser\_test Test}
\label{subsec:cfg66_tests_cliparser_test}

    This test uses the following tests options (only one is shown)
    static initialization:

    \begin{verbatim}
    static cfg::options::container s_test_options
    {
        //   Name, Code,  Kind, Enabled,
        //   Default, Value, FromCli, Dirty,
        //   Description, Built-in
        {
            {
                "alertable",
                {
                    'a', "boolean", cfg::options::enabled,
                    "false", "false", false, false,
                    "If specified, the application is alertable.",
                    false
                }
            }, . . .
        }
    };
    \end{verbatim}

    It serves as a good example of how to create a list of options.
    More flexible than GNU's getopt setup and simplifies generating help
    text.

    The test makes changes to the options and verifies that they
    took hold.
    The test command is not as simple as the \texttt{C} version, as
    verbosity is needed to see the changes:

    \begin{verbatim}
        $ ./build/test/cliparser_test --verbose
    \end{verbatim}

    It shows the changes and a result statement.
    This test needs a little bit of cleanup.

\subsection{Cfg66 history\_test Test}
\label{subsec:cfg66_tests_history_test}

    The history test also sets up a test options "array".
    Then it makes changes to the options, such as changing variables,
    undoing the change, and redoing the change.
    It shows the changes and a result statement.

    See this test file for some "To do" items.

\subsection{Cfg66 ini\_test Test}
\label{subsec:cfg66_tests_ini_test}

   This test program uses all of these "data" headers:

   \begin{itemize}
      \item texttt{ctrl\_spec.hpp}
      \item texttt{drums\_spec.hpp}
      \item texttt{mutes\_spec.hpp}
      \item texttt{palette\_spec.hpp}
      \item texttt{playlist\_spec.hpp}
      \item texttt{rc\_spec.hpp}
      \item texttt{session\_spec.hpp}
      \item texttt{usr\_spec.hpp}
    \end{itemize}

    It defines these static test items:

    \begin{itemize}
      \item \texttt{cfg::options::container s\_test\_options}.
         This sets up a single option called "test",
         used as a command-line option.
      \item \texttt{cfg::inisection::specification s\_simple\_ini\_spec}.
         This sets up an \texttt{[experiments]} section with a number of
         option-variable definitions.
      \item \texttt{cfg::inisection::specification s\_section\_spec}.
         This sets up an \texttt{[section-test]} section.
      \item \texttt{cfg::inifile::specification exp\_file\_data}
         This items sets up the "experiment" configuration directory
         using
         \texttt{cfg::inisection::specification s\_simple\_ini\_spec}.
         \texttt{cfg::inisection::specification s\_section\_spec}.
    \end{itemize}

    Additional sections are defined and add to a
    \texttt{cfg::inifile::specification} declaration.

    Hmmm. some are unsued.

\subsection{Cfg66 manager\_test Test}
\label{subsec:cfg66_tests_manager_test}

   This test defines 
   \texttt{cfg::appinfo s\_application\_info}.
   This is used here:
   \texttt{cfg::initialize\_appinfo(s\_application\_info, argv[0])}.

   The "To do" here is to actually implement \texttt{simple\_smoke\_test}.

\subsection{Cfg66 options\_test Test}
\label{subsec:cfg66_tests_options_test}

   This test program uses only the "data" header
   \texttt{test\_spec.hpp} which defines
   \texttt{cfg::options::container s\_test\_options}.
   This container is used to initialize a
   \texttt{cli::parser}.
   That object then gets the command-line arguments.

   Obviously, we still have a lot of work to do with these tests.

%-------------------------------------------------------------------------------
% vim: ts=3 sw=3 et ft=tex
%-------------------------------------------------------------------------------

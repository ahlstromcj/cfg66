%-------------------------------------------------------------------------------
% util
%-------------------------------------------------------------------------------
%
% \file        util.tex
% \library     Documents
% \author      Chris Ahlstrom
% \date        2024-04-16
% \update      2024-07-24
% \version     $Revision$
% \license     $XPC_GPL_LICENSE$
%
%     Provides a description of the entities in the cfg66 library, in the
%     util namespace.
%
%-------------------------------------------------------------------------------

\section{Util Namespace}
\label{sec:util_namespace}

   This section provides a useful walkthrough of the \texttt{util} namespace of
   the \textsl{cfg66} library.

   Here are the classes (or modules) in this namespace:

   \begin{itemize}
      \item \texttt{util::bytevector class}
      \item \texttt{util::filefunctions module}
      \item \texttt{util::msgfunctions module}
      \item \texttt{util::named\_bools class}
      \item \texttt{util::strfunctions module}
   \end{itemize}

   All of the modules are \texttt{C++} modules with free functions
   in the \texttt{util} namespace.

\subsection{util::bytevector}
\label{subsec:util_namespace_bytevector}

   The \texttt{util::bytevector} class
   provides an \texttt{std::vector} of "bytes" (\texttt{unsigned char}) with
   functions to put bytes into the vector and read them out.
   There are also functions to read a file and write the vector to the
   file.

   The bytes are treated as a stream of big-endian data.
   Integers are extracted from the bytes a byte at a time, starting with
   the most significant byte.
   Since this data is big-endian, it is suitable for use with MIDI
   files and network data streams.

   This module defines some types in the \texttt{util} namespace:

   \begin{itemize}
      \item \texttt{byte} = \texttt{uint8\_t}
      \item \texttt{ushort} = \texttt{uint16\_t}
      \item \texttt{ulong} = \texttt{uint32\_t}
      \item \texttt{ulonglong} = \texttt{uint64\_t}
      \item \texttt{bytes} = \texttt{std::vector<byte>}.
   \end{itemize}

   The \texttt{util::bytes} type holds the big-endian data.
   The \texttt{util::bytevector} class keeps track of the
   section of data being processed, and the position in the data.
   In addition, errors are detected, and an error message is stored
   for later display.

   Various overloads of the \texttt{assign} member function are used
   for loading data into the byte-vector.
   Functions are provided to "get" and "peek" at items in the byte-vector.
   The former change the position value, while the latter do not.
   Functions are provided to "put" and "poke" items into the byte-vector.
   The former change the position value, while the latter do not.

   Lastly, "read" and "write" functions are provided for interactions
   with a data file.

\subsection{util::filefunctions module}
\label{subsec:util_namespace_filefunctions}

   The \texttt{util::filefunctions} module contains a large number of
   function dealing file-names and files.
   The file-name functions are useful for building paths and for splitting
   paths into parts.
   The file functions are basically wrappers around the \texttt{C FILE *}
   API.
   Also provided are functions to check file attributes, to read/write
   strings, and to open, close, copy, and append data.

   Really, just skim the \texttt{filefunctions} modules to learn what is
   there.  They include every convenience function we needed in implementing
   \textsl{Seq66}.

\subsection{util::msgfunctions module}
\label{subsec:util_namespace_msgfunctions}

   The \texttt{util::msgfunctions} module defines functions for writing
   messages to the console along with tags showing the short name of the
   application that wrote them, and in color represent whether the message
   is an error, warning, debug, status, file-related, or session message.

   Also included are some "async safe" functions for output and for
   converting unsigned numbers to string arrays.

   Setters and getters are provided for the \texttt{--verbose} and
   \texttt{--investigate} options of the command-line parser.

   Again, skim the \texttt{util::msgfunctions} module for details.

\subsection{util::named\_bools}
\label{subsec:util_namespace_named_bools}

   The \texttt{util::name\_bools} class
   makes it easy to look up and set a "small" number of
   boolean values by name.

   This class could be useful if one does not want the full capability
   of the options-related classes in the \texttt{cfg} namespace.

\subsection{util::strfunctions module}
\label{subsec:util_namespace_strfunctions}

   The \texttt{util::strfunctions} module defines functions for manipulating
   strings: tokenization, left/right space trimming, conversion between
   strings and values with the added feature of defaulting, word-wrapping,
   and formatting of \texttt{std::string} values without using
   \texttt{std::stringstream}.

   Skim the \texttt{util::strfunctions} module for details.

%-------------------------------------------------------------------------------
% vim: ts=3 sw=3 et ft=tex
%-------------------------------------------------------------------------------

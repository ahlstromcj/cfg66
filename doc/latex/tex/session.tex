%-------------------------------------------------------------------------------
% session
%-------------------------------------------------------------------------------
%
% \file        session.tex
% \library     Documents
% \author      Chris Ahlstrom
% \date        2024-04-16
% \update      2024-04-16
% \version     $Revision$
% \license     $XPC_GPL_LICENSE$
%
%     Provides a description of the entities in the cfg66 library.
%
%-------------------------------------------------------------------------------

\section{Session Namespace}
\label{sec:session_namespace}

   This section provides a useful walkthrough of the \texttt{session} namespace of
   the \textsl{cfg66} library.
   In addition, a \texttt{C}-only module is provided.

   Here are the classes (or modules) in this namespace:

   \begin{itemize}
      \item \texttt{session::climanager}
      \item \texttt{session::configuration}
      \item \texttt{session::directories}
      \item \texttt{session::manager}
   \end{itemize}

\subsection{session::climanager}
\label{subsec:session_namespace_climanager}

   The \texttt{session::climanager} class is derived from
   \texttt{session::manager} and overrides a number of virtual functions
   It also provides a function to read a configuration file.
   It provides a \texttt{run()} loop which does nothing
   but check for calls to close and save the session and wait for a small
   polling period.

\subsection{session::configuration}
\label{subsec:session_namespace_configuration}

   The \texttt{session::configuration} class is derived from
   \texttt{cfg::basesettings}.
   It contains a \texttt{session::directories} management class
   and a set of
   \texttt{directories::entries} items.
   Options for help, a log-file, and auto-save are provided.

\subsection{session::directories}
\label{subsec:session_namespace_directories}

   The \texttt{session::directories} class manages a set of
   \texttt{entry} directory item.

   Each \texttt{entry} specifies:

   \begin{itemize}
      \item Name of the section covered by a configuration file.
      \item It active/inactive status.
      \item The directory for the files(s).
      \item The base name of the file(s).
      \item The optional extension of the file(s).
   \end{itemize}

   Provides the full path specification of each file, constructed
   from the entries, and keyed by the section name.
   The "home" configuration path and the session path are also specified.

   The \texttt{directories.cpp} module explains the directory layouts
   and provides default "rc" and "log" directory specifications.

\subsection{session::manager}
\label{subsec:session_namespace_manager}

   The \texttt{session::manager} class is base class for providing an
   application with "session" information, where a session is a group of
   configuration items that allow an application to run in a sequestered
   environment. Think of the \textsl{JACK Session Manager} or the 
   \textsl{New/Non Session Manager}.

   The base session manager class holds the following information:

   \begin{itemize}
      \item \texttt{session::configuration}.
         See the section about this class above.
      \item \texttt{cli::parser}.
         Provides access to the command-line parser.
      \item \textsl{Capabilities}.
         This application-dependent string publishes some information about
         the application. Useful with the
         \textsl{New/Non Session Manager}.
      \item \textsl{Manager name}.
         The name of the session manager. For example, it is returned by the
         \textsl{New/Non Session Manager}.
      \item \textsl{Manager path}.
         This item holds the directory where the session information is to
         be stored.
         For example, the \textsl{New/Non Session Manager} returns a string
         like
         \texttt{/home/user/NSM Sessions/JackSession/seq66.nUKIE}.
      \item \textsl{Display name}.
         This is the name of the session to be displayed, such as
         \textsl{JackSession} in the string above.
      \item \textsl{Client ID}.
         This is the name of the client (e.g. for managing port
         connections), such as \textsl{Seq66} or \textsl{seq66.nUKIE}.
   \end{itemize}

   Also included are indicators for \texttt{--help}, dirty status, and error
   messages.

   A large number of \texttt{virtual} members functions are included.
   Some of the important functions are for the following actions:

   \begin{itemize}
      \item Parsing an option (configuration) file.
      \item Parsing the command line.
      \item Creating, writing, and reading a configuration file.
      \item Creating, saving, and closing a session.
      \item Creating a session directory.
      \item Creating a "project".
      \item Creating a manager.
      \item Running a session, often in a loop or a GUI thread.
   \end{itemize}
   
   The \texttt{manager.cpp} module contains a short
   statically-initialized list of
   default options.

   Note that there are currently a number of "To Dos" in this class.

%-------------------------------------------------------------------------------
% vim: ts=3 sw=3 et ft=tex
%-------------------------------------------------------------------------------

%-------------------------------------------------------------------------------
% session
%-------------------------------------------------------------------------------
%
% \file        session.tex
% \library     Documents
% \author      Chris Ahlstrom
% \date        2024-04-16
% \update      2024-08-08
% \version     $Revision$
% \license     $XPC_GPL_LICENSE$
%
%     Provides a description of the entities in the cfg66 library.
%
%-------------------------------------------------------------------------------

\section{Sessions}
\label{sec:sessions}

\subsection{Session Description}
\label{subsec:session_description}

   A session provides a specific configuration for an application.
   The environment is contained and specified in a
   \textsl{session directory}, also known as a
   \textsl{home directory}.
   As an example, home directories for \textsl{Seq66} running on its
   own or inside an NSM session:

   \begin{verbatim}
      /home/user/.config/seq66
      /home/user/NSM Sessions/Seq66/seq66.nGJDW
   \end{verbatim}

   Note that the \texttt{seq66.nGJDW} directory is created by NSM.
   \textsl{Seq66} does not create a configuration directory by default, 
   but it has some options:

   \begin{itemize}
      \item \texttt{--home}.
         This option changes the "home" directory or appends a directory name
         to the default "home".
      \item \texttt{--session}.
         This option selects a setup name from \texttt{sessions.rc}.
         Each setup has a named section (e.g. \texttt[normal]) that specifies
         the "home" directory, the client name, base-name for the
         INI configuration files, and the name of a log file.
         This file is only read, never written.
         Since value names (e.g. "home") are duplicated, special processing
         is needed.
   \end{itemize}

   The "home" directory could have subdirectories, such as
   \texttt{config}, \texttt{audio}, \texttt{midi}, \texttt{data}, etc.

   Currently \textsl{Cfg66} provides a \texttt{.session} file that lists
   the names, locations, and activity of the various other INI files.
   With that setup, we would need a session file for each session.
   With NSM the session setup is directory-based.
   Use cases to support:

   \begin{itemize}
      \item \textbf{Standalone application}.
         The application automatically refers to a set of configuration files
         in a "home" directory,
         either \texttt{~/home/user/.config/seq66} or
         either \texttt{~/home/user/.config/seq66/config}.
      \item \textbf{Session selection}.
         All session information resides in a known file in the home
         directory, \texttt{sessions.rc}, and a selection of the session
         is made at startup.
      \item \textbf{NSM session}.
         There is a session manager that communicates the session directory
         to the application.
   \end{itemize}

\section{Session Namespace}
\label{sec:session_namespace}

   What is a session?
   It is an environment in which an application runs, and is one of many
   possible environments for the application.
   \textsl{Cfg66} provides rudimentary management of the characteristics of
   an application. Compare that to a true session manager such as the
   \textsl{New Session Manager}(\cite{nsm}, NSM), which coordinates not only
   the main application, but a group of applications and their configuration
   and connections, recreating recreate complex setups based on the
   contents of a particular session directory..
   Here, we want a way to provide various setups for an application.
   Of course, these setups can be placed under the control of a more complex
   session manager.

   This section provides a useful walkthrough of the \texttt{session} namespace of
   the \textsl{cfg66} library.
   In addition, a \texttt{C}-only module is provided.

   Here are the classes (or modules) in this namespace:

   \begin{itemize}
      \item \texttt{session::climanager}
      \item \texttt{session::configuration}
      \item \texttt{session::directories}
      \item \texttt{session::manager}
   \end{itemize}

\subsection{session::climanager}
\label{subsec:session_namespace_climanager}

   The \texttt{session::climanager} class is derived from
   \texttt{session::manager} and overrides a number of virtual functions
   It also provides a function to read a configuration file.
   It provides a \texttt{run()} loop which does nothing
   but check for calls to close and save the session and wait for a small
   polling period.

\subsection{session::configuration}
\label{subsec:session_namespace_configuration}

   The \texttt{session::configuration} class is derived from
   \texttt{cfg::basesettings}.
   It contains a \texttt{session::directories} management class
   and a set of
   \texttt{directories::entries} items.
   Options for help, a log-file, and auto-save are provided.

\subsection{session::directories}
\label{subsec:session_namespace_directories}

   The \texttt{session::directories} class manages a set of
   \texttt{entry} directory item.

   Each \texttt{entry} specifies:

   \begin{itemize}
      \item Name of the section covered by a configuration file.
      \item It active/inactive status.
      \item The directory for the files(s).
      \item The base name of the file(s).
      \item The optional extension of the file(s).
   \end{itemize}

   Provides the full path specification of each file, constructed
   from the entries, and keyed by the section name.
   The "home" configuration path and the session path are also specified.

   The \texttt{directories.cpp} module explains the directory layouts
   and provides default "rc" and "log" directory specifications.

\subsection{session::manager}
\label{subsec:session_namespace_manager}

   The \texttt{session::manager} class is base class for providing an
   application with "session" information, where a session is a group of
   configuration items that allow an application to run in a sequestered
   environment. Think of the \textsl{JACK Session Manager} or the 
   \textsl{New/Non Session Manager}.

   The base session manager class holds the following information:

   \begin{itemize}
      \item \texttt{session::configuration}.
         See the section about this class above.
      \item \texttt{cli::parser}.
         Provides access to the command-line parser.
      \item \textsl{Capabilities}.
         This application-dependent string publishes some information about
         the application. Useful with the
         \textsl{New/Non Session Manager}.
      \item \textsl{Manager name}.
         The name of the session manager. For example, it is returned by the
         \textsl{New/Non Session Manager}.
      \item \textsl{Manager path}.
         This item holds the directory where the session information is to
         be stored.
         For example, the \textsl{New/Non Session Manager} returns a string
         like
         \texttt{/home/user/NSM Sessions/JackSession/seq66.nUKIE}.
      \item \textsl{Display name}.
         This is the name of the session to be displayed, such as
         \textsl{JackSession} in the string above.
      \item \textsl{Client ID}.
         This is the name of the client (e.g. for managing port
         connections), such as \textsl{Seq66} or \textsl{seq66.nUKIE}.
   \end{itemize}

   Also included are indicators for \texttt{--help}, dirty status, and error
   messages.

   A large number of \texttt{virtual} members functions are included.
   Some of the important functions are for the following actions:

   \begin{itemize}
      \item Parsing an option (configuration) file.
      \item Parsing the command line.
      \item Creating, writing, and reading a configuration file.
      \item Creating, saving, and closing a session.
      \item Creating a session directory.
      \item Creating a "project".
      \item Creating a manager.
      \item Running a session, often in a loop or a GUI thread.
   \end{itemize}
   
   The \texttt{manager.cpp} module contains a short
   statically-initialized list of
   default options.

   Note that there are currently a number of "To Dos" in this class.

%-------------------------------------------------------------------------------
% vim: ts=3 sw=3 et ft=tex
%-------------------------------------------------------------------------------

%-------------------------------------------------------------------------------
% session
%-------------------------------------------------------------------------------
%
% \file        session.tex
% \library     Documents
% \author      Chris Ahlstrom
% \date        2024-04-16
% \update      2024-08-25
% \version     $Revision$
% \license     $XPC_GPL_LICENSE$
%
%     Provides a description of the entities in the cfg66 library.
%
%-------------------------------------------------------------------------------

\section{Sessions}
\label{sec:sessions}

\subsection{Session Naming Conventions}
\label{subsec:session_naming_conventions}

   The \textsl{Cfg66} session support relies heavily on directories.
   Here are the shorthand names we will use for them; "\$username" is the name
   of the user account; "\$appname" is the short name of the application;
   "\$clientname" is "\$appname" with a distinguishing wart (e.g \texttt{-123};
   "\$app-version" is "\$appname" with the version appended
   (e.g. \texttt{-0.99}.

   \begin{itemize}
      \item \texttt{\$prefix}.
         The base installation directory of the application that uses the
         \textsl{Cfg66} library.
         \begin{itemize}
            \item \textsl{Linux}:
               \texttt{/usr/} or \texttt{/usr/local}.
               Some common subdirectories:
               \begin{itemize}
                  \item \texttt{bin/}. 
                     Contains the application's executable file.
                  \item \texttt{include/\$app-version/}.
                     Contains the directory hierarchy of the application,
                     to support building the application from source code.
                  \item \texttt{lib/\$app-version/}.
                     Contains the libraries needed for the application.
                  \item \texttt{man/}. 
                     Contains the application's man pages.
                  \item \texttt{doc/\$app-version/}. 
                     Contains the application's documentation, which could
                     include HTML text, PDF files, tutorial files, and more.
                  \item \texttt{share/\$app-version/}. 
                     Contains stock data for the application, including
                     icons, pixmaps, sample data files, and sample
                     configuration files.
               \end{itemize}
            \item \textsl{BSD}:
               \texttt{/usr/local}.
                  \textsl{FreeBSD} reserves \texttt{/usr/} for system
                  files. Otherwise, the layout is about the same as for
                  \textsl{Linux}.
            \item \textsl{Windows}:
               The layout for \textsl{Windows} is notably different.
               \texttt{C:/Program Files/}. 
               The user can choose a disk drive other than \texttt{C:}.
               We currently aim to support only 64-bit applications.
               Here are the common directories as set up by \textsl{Cfg66}.
               \begin{itemize}
                  \item \texttt{C:/Program Files/\$appname/}. 
                     We might consider installing the executable into a
                     versioned subdirectory of this one.
                  \item \texttt{C:/Program Files/\$appname/data}. 
                     Contains all the installed items that are in
                     \texttt{share/} in UNIXen.
               \end{itemize}
         \end{itemize}
      \item \texttt{\textasciitilde}.
         The HOME directory of the user as supported in various operating
         systems.
         An equivalent symbol is \texttt{\$HOME}.
         \begin{itemize}
            \item \textsl{Linux}:
               \texttt{/home/username/}
            \item \textsl{BSD}:
               \texttt{/home/username/}
            \item \textsl{Windows}:
               \texttt{C:/Users/username/}
         \end{itemize}
      \item \texttt{\$home}.
         The directory of the configuration files as supported by default in
         \textsl{Cfg66}.
         It can be modified by a command-line option as described in the
         next section.
         \begin{itemize}
            \item \textsl{Linux}:
               \texttt{/home/username/.config/\$appname}.
            \item \textsl{BSD}:
               \texttt{/home/username/.config/\$appname}.
            \item \textsl{Windows}:
               \texttt{C:/Users/\$username/AppData/Local/\$appname}.
         \end{itemize}
      \item \texttt{\$directory}.
         This subdirectory can be specified in the \texttt{.session} file.
         Note that session files are \textsl{always} stored in
         \textsl{\$home}, even if \texttt{\$home} is modified by a
         command-line option.
         If \texttt{\$directory} is not empty,
         it replaces \texttt{\$home} as the home-directory of
         the application while it is running.
         It can have a full path if desired.
         If it has no path, then it is a subdirectory of \texttt{\$home}.
   \end{itemize}

   Confused? I sure am!

\subsection{Session Description}
\label{subsec:session_description}

   A session provides a specific configuration for an application.
   The environment is contained and specified in a
   \textsl{session directory}, also known as a
   \textsl{home directory}.
   As an example, home directories for \textsl{Seq66} running on its
   own or inside an \textsl{NSM} (\cite{nsm}) session:

   \begin{verbatim}
      /home/user/.config/seq66
      /home/user/NSM Sessions/Seq66/seq66.nGJDW
   \end{verbatim}

   Note that the \texttt{seq66.nGJDW} directory is created by \texttt{NSM}.
   \textsl{Cfg66} does not create a configuration directory by default, 
   but it has some options:

   \begin{itemize}
      \item \texttt{--home \textless path \textgreater}.
         This option changes the "home" directory if the argument includes 
         a path, or appends a directory name to the default "home" if it
         does not include a path.
         This changes the directory where the \texttt{.session} files are
         found.
         If \texttt{NSM} is running (as shown above), that directory
         becomes \texttt{\$home}.
      \item \texttt{--session \textless name[.session] \textgreater}.
         This option selects a session file from \texttt{\$home}.
         The default session file \texttt{\$appname.session}.
         This option is \textsl{not} to be used if \texttt{NSM} is running.
   \end{itemize}

   The "home" directory could have subdirectories, such as
   \texttt{config}, \texttt{audio}, \texttt{midi}, \texttt{data}, etc.
   These are application specific, and can be listed in the 
   \texttt{\$appname.session} file.

   \textsl{Cfg66} supports a \texttt{.session} file that lists
   the names, locations, and activity of the various other INI files.
   With that setup, we would need a session file for each session.
   With \texttt{NSM} the session setup is determined by
   \texttt{NSM}, and the session file is always \texttt{\$appname.session}.
   Use cases to support:

   \begin{itemize}
      \item \textbf{Standalone application}.
         The application automatically refers to a set of configuration files
         in a "home" directory,
         either \texttt{~/home/user/.config/seq66} or
         either \texttt{~/home/user/.config/seq66/config}.
      \item \textbf{Session selection}.
         All session information resides in a known file in the home
         directory, \texttt{\$appname.session}, but a selection of the session
         file can be made at startup.
      \item \textbf{NSM session}.
         There is a session manager that communicates the session directory
         to the application.
   \end{itemize}

\section{Session Namespace}
\label{sec:session_namespace}

   What is a session?
   It is an environment in which an application runs, and is one of many
   possible environments for the application.
   \textsl{Cfg66} provides rudimentary management of the characteristics of
   an application. Compare that to a true session manager such as the
   \textsl{New Session Manager}(\cite{nsm}, NSM), which coordinates not only
   the main application, but a group of applications and their configuration
   and connections, recreating recreate complex setups based on the
   contents of a particular session directory..
   Here, we want a way to provide various setups for an application.
   Of course, these setups can be placed under the control of a more complex
   session manager.

   This section provides a useful walkthrough of the \texttt{session} namespace of
   the \textsl{cfg66} library.
   In addition, a \texttt{C}-only module is provided.

   Here are the classes (or modules) in this namespace:

   \begin{itemize}
      \item \texttt{session::climanager}
      \item \texttt{session::configuration}
      \item \texttt{session::directories}
      \item \texttt{session::manager}
   \end{itemize}

\subsection{session::climanager}
\label{subsec:session_namespace_climanager}

   The \texttt{session::climanager} class is derived from
   \texttt{session::manager} and overrides a number of virtual functions
   It also provides a function to read a configuration file.
   It provides a \texttt{run()} loop which does nothing
   but check for calls to close and save the session and wait for a small
   polling period.

\subsection{session::configuration}
\label{subsec:session_namespace_configuration}

   The \texttt{session::configuration} class is derived from
   \texttt{cfg::basesettings}.
   It contains a \texttt{session::directories} management class
   and a set of
   \texttt{directories::entries} items.
   Options for help, a log-file, and auto-save are provided.

   Actually, using the \texttt{cfg::inimanager} seems like a better option.
   Stay tuned.

\subsection{session::directories}
\label{subsec:session_namespace_directories}

   The \texttt{session::directories} class manages a set of
   \texttt{entry} directory item.

   Each \texttt{entry} specifies:

   \begin{itemize}
      \item Name of the section covered by a configuration file.
      \item It active/inactive status.
      \item The directory for the files(s).
      \item The base name of the file(s).
      \item The optional extension of the file(s).
   \end{itemize}

   Provides the full path specification of each file, constructed
   from the entries, and keyed by the section name.
   The "home" configuration path and the session path are also specified.

   The \texttt{directories.cpp} module explains the directory layouts
   and provides default "rc" and "log" directory specifications.

\subsection{session::manager}
\label{subsec:session_namespace_manager}

   The \texttt{session::manager} class is base class for providing an
   application with "session" information, where a session is a group of
   configuration items that allow an application to run in a sequestered
   environment. Think of the \textsl{JACK Session Manager} or the 
   \textsl{New/Non Session Manager}.

   The base session manager class holds the following information:

   \begin{itemize}
      \item \texttt{session::configuration}.
         See the section about this class above.
      \item \texttt{cli::parser}.
         Provides access to the command-line parser.
      \item \textsl{Capabilities}.
         This application-dependent string publishes some information about
         the application. Useful with the
         \textsl{New/Non Session Manager}.
      \item \textsl{Manager name}.
         The name of the session manager. For example, it is returned by the
         \textsl{New/Non Session Manager}.
      \item \textsl{Manager path}.
         This item holds the directory where the session information is to
         be stored.
         For example, the \textsl{New/Non Session Manager} returns a string
         like
         \texttt{/home/user/NSM Sessions/JackSession/seq66.nUKIE}.
      \item \textsl{Display name}.
         This is the name of the session to be displayed, such as
         \textsl{JackSession} in the string above.
      \item \textsl{Client ID}.
         This is the name of the client (e.g. for managing port
         connections), such as \textsl{Seq66} or \textsl{seq66.nUKIE}.
   \end{itemize}

   Also included are indicators for \texttt{--help}, dirty status, and error
   messages.

   A large number of \texttt{virtual} members functions are included.
   Some of the important functions are for the following actions:

   \begin{itemize}
      \item Parsing an option (configuration) file.
      \item Parsing the command line.
      \item Creating, writing, and reading a configuration file.
      \item Creating, saving, and closing a session.
      \item Creating a session directory.
      \item Creating a "project".
      \item Creating a manager.
      \item Running a session, often in a loop or a GUI thread.
   \end{itemize}
   
   The \texttt{manager.cpp} module contains a short
   statically-initialized list of
   default options.

   Note that there are currently a number of "To Dos" in this class.

\section{Session Walkthroughs}
\label{sec:session_walkthroughs}

	While the session layouts that can be supported are many, we will
	walk through only the most common (i.e. tested) scenarios.

   \begin{enumber}
      \item Determine the \texttt{\$home} directory.
      \begin{enumber}
         \item Provide a \texttt{cfg::appinfo} structure
            that defines at least the app-name.  One can also provide
            non-empty string values for app-version, config-section name,
            home directory and home configuration file, etc. If empty,
            defaults are used or assembled.
         \item Optionally, one can call \texttt{user\_home(appfolder)}
            to change the folder name at run-time.
         \item Another option is to use the \texttt{--home path} option
            on the command-line.
         \item If a session manager is running, use the path that it
            provides.
      \end{enumber}
      \item Determine the session file.
         By default this is \texttt{\$appname.session} in the \texttt{\$home}
         directory. It can be changed only with the \texttt{--session}
         command-line option.
      \item Determine the session \texttt{\$home} directory.
         By default this is the same as \texttt{\$home}.
      \item Get the list of application sub-directories (for data and
         other files) from the session file.
         Make sure each directory is created or already in existence.
   \end{enumber}

   Note that \texttt{\$home} can be set in the beginning via the CLI or
   function call, but with a session manager it cannot be determined until
   after a connection is made.
   Therefore, the \texttt{manager::run()} override must start with a call to
   \texttt{set\_home()}.

   TODO.

   (We need a special inifile class to process a .session file.

   TODO.
   TODO.
   TODO.

%-------------------------------------------------------------------------------
% vim: ts=3 sw=3 et ft=tex
%-------------------------------------------------------------------------------

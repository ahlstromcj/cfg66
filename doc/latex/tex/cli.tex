%-------------------------------------------------------------------------------
% cli
%-------------------------------------------------------------------------------
%
% \file        cli.tex
% \library     Documents
% \author      Chris Ahlstrom
% \date        2024-04-16
% \update      2024-07-20
% \version     $Revision$
% \license     $XPC_GPL_LICENSE$
%
%     Provides a description of the entities in the cfg66 library.
%
%-------------------------------------------------------------------------------

\section{Cli Namespace}
\label{sec:cli_namespace}

   This section provides a useful walkthrough of the \texttt{cli} namespace of
   the \textsl{cfg66} library.
   In addition, a \texttt{C}-only module is provided.

   Here are the classes (or modules) in this namespace:

   \begin{itemize}
      \item \texttt{cliparser\_c}
      \item \texttt{cli::parser}
      \item \texttt{cli::multiparser}
   \end{itemize}

\subsection{cliparser\_c Module}
\label{subsec:cli_namespace_c}

   This module is actually a \texttt{C++} module that implements a number of
   \texttt{extern "C"} functions.
   The functions themselves access an internal and hidden
   \texttt{cli::parser} object and call its member functions to
   perform the functions.

   This module provides a \texttt{C} structure that mirrors
   \texttt{options::spec}, plus some free functions to access this
   structure.

\subsection{cli::parser}
\label{subsec:cli_namespace_parser}

   The \texttt{cli::parser} class contains a number of options in a
   \texttt{cfg::options} object.
   Many of the member functions are pass-alongs to this object.
   This parser is meant for simple usage; it has no support for accessing
   configuration files, and supports just one set of options.

\subsection{cli::parser / Global Options}
\label{subsec:cli_namespace_parser_global_options}

   The \texttt{cli::parser} also provides support for options that can
   be command to all applications:

   \begin{itemize}
      \item \texttt{--version}.
         Provides the version of the application.
      \item \texttt{--help}.
         Emits help text. This help text is assembled from
         the specification structures set up according to the
         \texttt{cfg::options::container}.
      \item \texttt{--description}.
         This option is meant to show the description fields of the
         options.
      \item \texttt{--verbose}.
         Indicates to emit additional information about the run of the
         application.
      \item \texttt{--inspect}.
         This option is meant to activate debugging code.
         The applications determines what this means.
      \item \texttt{--investigate}.
         This option is meant to activate temporary debugging code for
         the current purpose desired by the programmer.
         The applications determines what this means.
      \item \texttt{--log}.
         This option enables a log file for recording error and information
         output.
   \end{itemize}

   The \texttt{parse()} function looks for stock option such as
   \texttt{--help} and \texttt{--option log filename}.
   The \texttt{--} sequence terminates a list of options.

   It is meant to be similar to \texttt{getopt(3)}, but much more flexible and
   perhaps easier to set up.

   The caller can call the following member functions:

   \begin{itemize}
      \item \texttt{show\_information\_only()}.
         This function returns a boolean value that is true if
         help, description, or version were requested.
         Generally, if this information is shown, the application does
         nothing but show the desired information, and exits.
      \item \texttt{version\_request()}.
         Indicates if version information was requested.
      \item \texttt{help\_request()}.
         Indicates if application help was requested.
         The command-line help text is shown.
         The caller can also provide further help if desired.
      \item \texttt{description\_request()}.
         Indicates if description information was requested.
      \item \texttt{verbose\_request()} and \texttt{verbose()}.
         Indicates if extra output (either command line or via
         dialog boxes) was requested.
      \item \texttt{inspect\_request()}.
         Indicates if inspection was specified was requested.
         Generally can be used to output data values as appropriate.
      \item \texttt{investigate\_request()}.
         Indicates if investigation was specified was requested.
         Defined by the application.
      \item \texttt{use\_log\_file()} and \texttt{log\_file()}.
         Indicates that standard I/O should be redirected to a log file.
   \end{itemize}

   There are functions to add, verify, and set option values:

   \begin{itemize}
      \item \texttt{add()}.
      \item \texttt{verify()}.
      \item \texttt{set\_value()}.
      \item \texttt{change\_value()}.
      \item \texttt{clear()}.
      \item \texttt{unmodify()}.
      \item \texttt{unmodify\_all()}.
   \end{itemize}

   There are also functions to test and retrieve option values:

   \begin{itemize}
      \item \textsl{string}.
         All options are encoded as strings.
         Strings can be retrieved using \texttt{cli::parser::value()}.
         The default string can be retrieved by \texttt{default\_value()}.
      \item \textsl{boolean}.
         \texttt{is\_boolean()} tests for a value being boolean.
      \item \textsl{other, to do}.
         There are many more value-related functions in the
         \texttt{cfg::options} class, but they use a reference to
         a \texttt{cfg::options::spec} structure, and can be accessed
         only by getting the option and spec reference.
         See \sectionref{subsec:cfg_namespace_options}.
   \end{itemize}

   The format of the default value is either a simple string showing the
   default value, or a \textsl{range} string such as "0<=10<=99", where
   the middle value is the default value.

\subsection{cli::parser / Usage}
\label{subsec:cli_namespace_parser_usage}

   This section provides a walk through the test application
   \texttt{cliparser\_test}. It starts with an \texttt{appinfo} function
   to see the name that will appear in console messages:

   \begin{verbatim}
      cfg::set_client_name("cli");
      cfg::set_app_version("0.2.0");
   \end{verbatim}

   Next, a parser is created via:

   \begin{verbatim}
      cli::parser clip(s_test_options);
   \end{verbatim}

   The \texttt{s\_test\_options} list is defined in the \texttt{test\_spec}
   header file. Here is an excerpt. Note that we could use an anonymous
   namespace instead of the keyword \texttt{static}.

   \begin{verbatim}
      static cfg::options::container s_test_options
      {
          {
              {
                  "alertable",
                  {
                      'a', cfg::options::kind::boolean, cfg::options::enabled,
                      "false", "", false, false,
                      "If specified, the application is alertable.", false
                  }
              },
              {
                  "canned-code",
                  {
                      'c', cfg::options::kind::boolean, cfg::options::enabled,
                      "true", "", false, false,
                      "If specified, the application employs canned code.", false
                  }
              },
              . . .
          }
      };
   \end{verbatim}

   As an example, this list provides the \texttt{-a} and \texttt{--alertable}
   option, which is a boolean option and is enabled. It has a default value
   of "false", which is set once the list is initialized.
   The values for "set-from-cli" and "dirty" are false by default at first.
   After the one-line description, the boolean indicates if the options
   is a built-in (global, stock) option. 
   That value is set only in the \texttt{global\_options()} function
   in the \texttt{options} module.
   The options in this list are added to the global options.

   After construction, the parser can be applied to the command-line
   arguments:

   \begin{verbatim}
       bool success = clip.parse(argc, argv);
   \end{verbatim}

   This can modify (set the value, raise the "set-from-cli" and "dirty"
   flags) options based on the command-line.

   Not all options will have a code, in general, especially if there are a
   large number of options. Duplicates are checked for.
   The existing codes can be listed:

   \begin{verbatim}
      std::string msg = "Option codes: " + clip.code_list();
   \end{verbatim}

   The command line can also be scanned to see if an option is present.
   The last parameter indicates if the option is required to exist:

   \begin{verbatim}
      bool findme_active = clip.check_option(argc, argv, "find-me", false);
   \end{verbatim}

   Built-in (global) options that make the application show something and
   then quit can be tested:

   \begin{verbatim}
      if (clip.show_information_only())
      {
          if (clip.help_request()) . . .
          if (clip.description_request()) . . .
          if (clip.version_request()) . . .
      }
   \end{verbatim}

   Some other built-in options:

   \begin{verbatim}
      if (clip.verbose_request()) . . .
      if (clip.inspect_request()) . . .
      if (clip.investigate_request()) . . .
      if (clip.use_log_file()) . . .
   \end{verbatim}

   Debug text can be shown; it lists all the options, their values, their
   default values, and if the value has been modified:

   \begin{verbatim}
      std::string dbgtxt = clip.debug_text(cfg::options::stock);
   \end{verbatim}

   Values can also be set or changed by the application itself, rather
   than from the command-line. It can be retrieved by the long option
   name or the option code.

   \begin{verbatim}
      success = clip.change_value("username", "C. Ahlstrom");
      std::string name = clip.value("u");    // or "username"
      success = name == "C. Ahlstrom";
   \end{verbatim}

   If an error occurs, which can be checked by a function's return code
   or by a call to \texttt{cli::parser::error\_msg()}, then the message can
   be retrieved for display:

   \begin{verbatim}
        std::string errmsg = clip.error_msg();
        util::error_message(errmsg);
   \end{verbatim}

\subsection{cli::multiparser}
\label{subsec:cli_namespace_multiparser}

   The \texttt{cli::multiparser} class extends \texttt{cli::parser} in
   a number of ways, in order to support a suite of command-line options
   covering mutltiple configuration files.

   Support is provided to look up the long option name from an option code
   character. (The \texttt{cfg::options} class also provides this, but it
   is not directly exposed to \texttt{cli::parser}.

   \begin{verbatim}
      using codes = std::map<char, std::string>;
      codes & code_mappings();
   \end{verbatim}

   This function is used in the \texttt{multiparser} override of the
   the virtual \texttt{cli::parser::parse()} function.

   In order to support multiple configuration files and multiple
   configuration sections, we need to way to find out in which file and section
   an option resides.

   \begin{verbatim}
      using duo = struct
      {
         std::string config_type;
         std::string config_section;
      };
      using names = std::map<std::string, duo>;
      names & cli_mappings();
   \end{verbatim}

   This function is used to find the desired option set, a pointer to
   the \texttt{cfg::options} for a particular file and section.

   For a walk-through, see the section concerning the "INI set test".
   TO DO.


%-------------------------------------------------------------------------------
% vim: ts=3 sw=3 et ft=tex
%-------------------------------------------------------------------------------

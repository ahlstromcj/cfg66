%-------------------------------------------------------------------------------
% cli
%-------------------------------------------------------------------------------
%
% \file        cli.tex
% \library     Documents
% \author      Chris Ahlstrom
% \date        2024-04-16
% \update      2024-04-16
% \version     $Revision$
% \license     $XPC_GPL_LICENSE$
%
%     Provides a description of the entities in the cfg66 library.
%
%-------------------------------------------------------------------------------

\section{Cli Namespace}
\label{sec:cli_namespace}

   This section provides a useful walkthrough of the \texttt{cli} namespace of
   the \textsl{cfg66} library.
   In addition, a \texttt{C}-only module is provided.

   Here are the classes (or modules) in this namespace:

   \begin{itemize}
      \item \texttt{cliparser\_c}
      \item \texttt{cli::parser}
   \end{itemize}

\subsection{cliparser\_c Module}
\label{subsec:cli_namespace_c}

   This module is actually a \texttt{C++} module that implements a number of
   \texttt{extern "C"} functions.
   The functions themselves access an internal and hidden
   \texttt{cli::parser} object and call its member functions to
   perform the functions.

   This module provides a \texttt{C} structure that mirrors
   \texttt{options::spec}, plus some free functions to access this
   structure.

\subsection{cli::parser}
\label{subsec:cli_namespace_parser}

   The \texttt{cli::parser} class contains a number of options in a
   \texttt{cfg::options} object.
   Many of the member functions are pass-alongs to this object.

   It also holds values indicating if some basic options (help, version,
   verbosity, log-file usage) are set.
   The \texttt{parse()} function looks for stock option such as
   \texttt{--help} and \texttt{--option log filename}.
   The \texttt{--} sequence can terminate a list of options.

   It is meant to be similar to getopt, but much more flexible and
   perhaps easier to set up.

%-------------------------------------------------------------------------------
% vim: ts=3 sw=3 et ft=tex
%-------------------------------------------------------------------------------

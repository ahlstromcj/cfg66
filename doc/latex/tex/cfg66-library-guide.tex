%-------------------------------------------------------------------------------
% cfg66-reference-guide
%-------------------------------------------------------------------------------
%
% \file        cfg66-reference-guide.tex
% \library     Documents
% \author      Chris Ahlstrom
% \date        2024-04-15
% \update      2024-04-17
% \version     $Revision$
% \license     $XPC_GPL_LICENSE$
%
%     This document provides LaTeX documentation for the cfg66 library.
%
%-------------------------------------------------------------------------------

\documentclass[
 11pt,
 twoside,
 a4paper,
 final                                 % versus draft
]{article}

\input{tex/docs-structure}             % specifies document structure and layout

\usepackage{fancyhdr}
\pagestyle{fancy}
\fancyhead{}
\fancyfoot{}
\fancyheadoffset{0.005\textwidth}
\lhead{Cfg66 Library Guide}
\chead{}
\rhead{Developer Guide}
\lfoot{}
\cfoot{\thepage}
\rfoot{}

% Removes the many "headheight is too small" warnings.

\setlength{\headheight}{14.0pt}

\makeindex

\begin{document}

\title{Cfg66 Developer Guide 0.1.0}
\author{Chris Ahlstrom \\
   (\texttt{ahlstromcj@gmail.com})}
\date{\today}
\maketitle

\begin{figure}[H]
   \centering 
   \includegraphics[scale=0.25]{cfg66.png}
   \caption*{Cfg66 Logo}
\end{figure}

\clearpage                             % moves Contents to next page

\tableofcontents
\listoffigures                         % print the list of figures
\listoftables                          % print the list of tables

% changes the paragraph style to remove indenting and put a line between each
% paragraph.  this could be moved up into the preamble, but then would
% affect the spacing of the toc and lof, lot noted above.

\setlength{\parindent}{2em}
\setlength{\parskip}{1ex plus 0.5ex minus 0.2ex}

\rhead{\rightmark}         % shows section number and section name

\section{Introduction}
\label{sec:introduction}

   The \textsl{Cfg66} library reworks some of the fundamental code
   from the \textsl{Seq66} project (\cite{seq66}).
   This work is in preparation for the version 2 of that project, but
   might also be useful in other applications.

   Cfg66 contains the following subdirectories of \texttt{src} and
   \texttt{include}, each of which holds modules in a
   namespace of the same name:

   \begin{itemize}
      \item \texttt{cfg}.
         Contains items that can be used to manage a generic
         configuration, including application names, settings basics,
         and an INI-style configuration-file system. Added are
         data-type indicators and help text.
      \item \texttt{cli}.
         Provides C/C++ code to handle command-line parsing without
         needing to use, for example, getopt. While it somewhat matches
         how getopt works, it also allows combining option sets and
         provides a parser object that contains the current status
         of all available options, as well as help text.
      \item \texttt{session}.
         Contains classes for managing a basic "session". Here, a
         session is simply a location to put configuration files;
         multiple locations can be supported. Session filenames are
         based on a "home" configuration directory, optional
         subdirectories, and application-specific item names.
      \item \texttt{util}.
         Contains file functions, message functions, string functions,
         and other functionality common to all our "66" application
         and libraries.
   \end{itemize}

   In the sections that follow, the basic are described.
   At some point we will make the effort to add some \textsl{Dia}
   diagrams to make the relationships more clear.

\subsection{Naming Conventions}
\label{subsec:introduction_conventions}

   \textsl{Cfg66} uses some conventions for naming things in this
   document.

   \begin{itemize}
      \item \texttt{\$prefix}. The base location for installation of
         the application and its ancillary data files on
         \textsl{UNIX/Linux/BSD}:
         \begin{itemize}
            \item \texttt{/usr/}
            \item \texttt{/usr/local/}
         \end{itemize}
      \item \texttt{\$winprefix}. The base location for installation of
         the application and its ancillary data files on \textsl{Windows}.
         \begin{itemize}
            \item \texttt{C:/Program Files/}
            \item \texttt{C:/Program Files (x86)/}
         \end{itemize}
      \item \texttt{\$home}. The location of the user's configuration files.
         Not to be confused with \texttt{\$HOME}, this is
         the standard location for configuration files.
         On a UNIX-style system, it would be \linebreak
         \texttt{\$HOME/.config/appname}.
         The files would be put into a \texttt{po} subdirectory here.
      \item \texttt{\$winhome}. This location is different for
         \textsl{Windows}:
         \texttt{C:/Users/user/AppData/Local/PACKAGE}.
   \end{itemize}

\subsection{Future Work}
\label{subsec:introduction_future}

   \begin{itemize}
      \item Hammer on this code in \textsl{Windows}.
   \end{itemize}

%----------------------------------------------------------------------------
% Additional Chapters
%----------------------------------------------------------------------------

%-------------------------------------------------------------------------------
% cfg
%-------------------------------------------------------------------------------
%
% \file        cfg.tex
% \library     Documents
% \author      Chris Ahlstrom
% \date        2024-04-16
% \update      2024-07-29
% \version     $Revision$
% \license     $XPC_GPL_LICENSE$
%
%     Provides a description of the entities in the cfg66 library, in the
%     "cfg" namespace.
%
%-------------------------------------------------------------------------------

\section{Cfg Namespace}
\label{sec:cfg_namespace}

   This section provides a useful walkthrough of the \texttt{cfg} namespace of
   the \textsl{cfg66} library.
   Here are the classes (or modules) in this namespace:

   \begin{itemize}
      \item \texttt{cfg::appinfo}
      \item \texttt{cfg::basesettings}
      \item \texttt{cfg::comments}
      \item \texttt{cfg::configfile}
      \item \texttt{cfg::history}
      \item \texttt{cfg::inifile}
      \item \texttt{cfg::inimanager}
      \item \texttt{cfg::inisection}
      \item \texttt{cfg::inisections}
      \item \texttt{cfg::memento}
      \item \texttt{cfg::options}
      \item \texttt{cfg::palette}
      \item \texttt{cfg::recent}
   \end{itemize}

\subsection{cfg::appinfo}
\label{subsec:cfg_namespace_appinfo}

   The \texttt{cfg::appinfo} structure encapsulates basic information about
   an application:

   \begin{itemize}
      \item \textsl{cfg::appkind}.
         Indicates if the application is a headless application (such as a
         daemon), a command-line, \textsl{ncurses}, GUI, or test application.
         In some cases it can be useful to know how the application is
         running.
      \item \textsl{App name}.
         Provides the short name of the application, which can be shown
         in warning messages or be used to name the application as a node, for
         example, in \textsl{JACK}.
      \item \textsl{App version}.
         Provides a version number, or optionally, a variant on the
         \textsl{Name} plus the version number.
      \item \textsl{Main config section name}.
         Holds the name of main configuration section (usually present
         only in the "session" or "rc" file).
         The default value is "[Cfg66]".
      \item \textsl{App home configuration directory}.
         Provides the name of the configuration directory for the application,
         such as \texttt{/home/user/.config/appname}.
         Modified by the \texttt{--home} option.
      \item \textsl{Home configuration file}.
         Provides the name of the main configuration file, such as
         \texttt{appname.rc}.
         Modified by the \texttt{--config} option.
      \item \textsl{Client name}.
         Either the short application name or a variant of it, useful in session
         managers, for example.
         An example under \textsl{NSM} would be \texttt{seq66v2.nUKIE}.
      \item \textsl{App tag}.
         The short application name with the version number tacked on.
      \item \textsl{Arg0}.
         Holds the complete path to the executable as determined at run-time.
      \item \textsl{Package name}.
         Provides the name of the package, which could be the same as the
         application or library name.
      \item \textsl{Session Tag}.
         Provides the session name for the application; it can be the
         application name, a modification of the application name, or
         something completely different, such as a session name given by
         a user.
         Useful in long error/warning/info messages.
      \item \textsl{App icon}.
         Provides the base name of the application icon, if not empty.
      \item \textsl{Version text}.
         Reconstructed version information for the application.
      \item \textsl{API engine}.
         Some applications might use various libraries.
         For example, for a MIDI application it might be one of these
         MIDI libraries:
         \texttt{rtmidi},
         \texttt{rtl66},
         or \texttt{portmidi}.
      \item \textsl{API version}.
         Indicates which version of an API is in force.
         In some cases this can be detected at run-time.
      \item \textsl{GUI version}.
         Indicates the GUI or "curses" version, such as \textsl{Qt 6.1}
         or \textsl{Gtkmm 3.0}.
      \item \textsl{Short client name}.
         Similar to \textsl{client name}, but never has anything appended
         to it.
      \item \textsl{Client name tag}.
         Provides the name to show on the console in error/warning/info messages,
         the short client name surrounded by brackets, such as
         \texttt{[seq66v2]}.
   \end{itemize}

   All of these items can be set at once using the \texttt{appinfo}
   constructor that has a large number of parameters.
   In addition, there are a many "free" functions in the \texttt{cfg}
   namespace for setting and getting these values.
   See \texttt{appinfo.hpp} for a summary.

\subsection{cfg::basesettings}
\label{subsec:cfg_namespace_basesettings}

   THe \texttt{cfg::basesettings} class is a base class.
   It provides common settings useful in any application:

   \begin{itemize}
      \item \textsl{File name}.
         Holds the (optional) name of the file that holds the settings data.
      \item \textsl{Ordinal version}.
         Provides a simple integer indicating the version of the file,
         useful in adapting to changes in settings.
      \item \textsl{Modify/unmodify function and "modified" flag}.
         Indicates if the settings have been modified.
      \item \textsl{Config format}.
         A free-form string indicating the format of the data, such as
         \texttt{INI}, \texttt{XML}, or \texttt{JSON}.
         For "66" libraries and applications, the \texttt{INI}
         format is sufficient.
      \item \textsl{Config type}.
         A short string that indicates something about the
         format or content of the file. For example, in Seq66,
         common values were 'rc' and 'usr', with these
         values also representing the file extension.
      \item \textsl{Ordinal version}.
         A simple integer that is incremented each time a change is
         made in the configuration format or values-present
         that isn't detectable during parsing.
      \item \textsl{Comments}.
         An object holding the main comments that describe something
         about the settings file.
      \item \textsl{Is error}.
         Indicated if there was a error during parsing.
      \item \textsl{Error message}.
         If an error occurred, the error message is stored for display.
   \end{itemize}

\subsection{cfg::comments}
\label{subsec:cfg_namespace_comments}

   \texttt{cfg::comments} holds a string describing
   something general about the configuration,
   meant to be included as the text of a \texttt{[comment]} section.
   Provides some setter and getter functions.
   It is a simple class.

\subsection{cfg::configfile}
\label{subsec:cfg_namespace_configfile}

   The \texttt{cfg::configfile} class is an \texttt{abstract base class}.
   It provides some items common to configuration files, including
   an extensive set of functions to parse sections and configuration
   variables in an INI format in a line-by-line fashion.
   The member functions \texttt{parse()} and \texttt{write()}
   are \textsl{pure virtual}, and must be overridden in a derived class.
   A good example is \texttt{cfg::inifile}.

   These are the main externally-accessible values:

   \begin{itemize}
      \item \textsl{File extension}.
         Common values are 'rc' and 'session'.
      \item \textsl{File name}.
         Provides the file name, normally as a full-path file-specification.
      \item \textsl{File version}.
         Provides the current version of the derived configuration file format.
         Set in the constructor of the configfile-derived object,
         and incremented in that object whenever a new way of reading,
         writing, or formatting the configuration file is created.
   \end{itemize}

   Also too many to list, but it includes functions such as
   \texttt{get\_integer()} and \texttt{write\_integer()}.

   These work by having the whole file read into an
   \texttt{std::ifstream}, and then searching the string over and over
   to read all the variables.
   Sounds inefficient, but in practise it is very fast.

   Finally, there are free functions to delete a configuration file and
   to make a copy of a configuration file.

\subsection{cfg::history}
\label{subsec:cfg_namespace_history}

   One of the things not handled so well in \textsl{Seq66} is the
   undo/redo functionality.
   The \texttt{history} template class implements undo/redo using the 
   \texttt{memento} class described below. It follows
   the \textsl{Design Patterns} book (\cite{dpatterns}).
   Also informative is \cite{deque}.
   Also see the \texttt{cfg::memento} class below.

   The heart of the \texttt{history} template is the history list:

   \begin{verbatim}
      std::deque<memento<TYPE>> m_history_list;
   \end{verbatim}

   Member functions are provided to see if history entries are
   undoable/redoable, undo and redo them, check the maximum size of the list,
   etc.
   The \texttt{history.cpp} module provide a small test of history for
   the \texttt{cfg::options} class.

   \index{history\_test}
   The \texttt{history\_test} application provides a more substantive
   test of history.
   It provides an \texttt{cfg::options::container} with a few options, and
   then applies \texttt{undo} and
   \texttt{redo} operations on them, checking the results.

   \textbf{To do}: test range validation.

\subsection{cfg::inifile}
\label{subsec:cfg_namespace_inifile}

   \texttt{cfg::inifile} is derived from the \texttt{cfg::configfile} class.
   It contains a reference to a class that manages multiple
   INI sections: \texttt{cfg::inisections}.
   It overrides the
   \texttt{cfg::configfile::parse()} and
   \texttt{cfg::configfile::write()} functions.
   It provides two protected functions,
   \texttt{parse\_section()} and
   \texttt{write\_section()}.

   \index{ini\_test}
   The \texttt{ini\_test} application sets up a couple of
   \texttt{cfg::inisection::specification} named structures 
   with a long description and a list of values to be stored.
   The \texttt{cfg::stock\_cfg66\_data()} and
   \texttt{cfg::stock\_comment\_data()} specification are defined as well.
   All section specifications are wrapped in a
   \texttt{cfg::inisections::specification} structure
   that is used to create a
   \texttt{cfg::inisections} object and then use it to
   create an INI file for output, exercising \texttt{inifile::write()}.
   The file generated is \texttt{tests/data/fooout.rc}.

   The same setup is used to exercise \texttt{inifile::parse()}.
   The file read is \texttt{tests/data/fooin.rc}.
   The file generated from \texttt{fooin.rc} is
   \texttt{tests/data/fooinout.rc}.
   The file generated from the test's internal data is
   \texttt{tests/data/fooout.rc}.
   The results can be found in the \texttt{tests/data} directory.
   The files \texttt{fooin.rc} and 
   \texttt{fooinout.rc} should be almost identical.

   How does it work?

   \begin{verbatim}
       cfg::inisections sections(exp_file_data);
   \end{verbatim}

   The \texttt{exp\_file\_data} specification provides
   the configuration type (file extension without the period) "exp",
   the putative configuration directory "~/.config/experiment/",
   the base name of the configuration file "ini\_test\_session",
   followed by the description of the INI sections object and
   a list of the sections in the INI sections object.

   TODO

\subsection{cfg::inimanager}
\label{subsec:cfg_namespace_inimanager}
         
   \texttt{cfg::inimanager} is meant to manage multiple
   \texttt{cfg::inifile} objects.
   Each INI file is associated with a unique configuration file type,
   such as "rc" or "session", and a section name,

   \texttt{cfg::inimanager} defines two types:

   \begin{itemize}
      \item \texttt{optionref}.
         \texttt{std::reference\_wrapper<options::spec>}.
      \item \texttt{optionmap}.
         \texttt{std::map<std::string, optionref>}.
   \end{itemize}

   Also defined is a function to add an named \texttt{cfg::option}
   to the option map.

   \index{ini\_set\_test}
   The \texttt{ini\_set\_test} application first defines some
   \textsl{global} options
   peculiar to it: "list", "read", "test", and "write".
   These are \texttt{global} options because they are not associated with
   an INI file or an INI section.
   The \texttt{cfg::inimanager} is created with this list, and populates
   the options. Then two \texttt{cfg::inisection} objects are
   added, one for \texttt{cfg::small\_data} and
   one for \texttt{cfg::rc\_data}.

   The caller can then exercise the global options (including the internal,
   stock options).  The \texttt{--help} option will show all of the global
   option, plus all of the sections and options defined for
   \texttt{cfg::small\_data} and
   \texttt{cfg::rc\_data}.
   The \texttt{--read} and \texttt{write} options accept a file-name and
   process it.

   Note: if something like the following message appears, fix the issue.
   Otherwise, the option cannot be parsed, even if it appears in the
   help text.

   \begin{verbatim}
      [iniset] Couldn't insert <sets-mode,(rc,[misc])>: Change option to a
      unique name
   \end{verbatim}

   Let us walk through some runs of
   \texttt{ini\_set\_test}.

   \begin{verbatim}
      $ ./build/test/ini_set_test --help
   \end{verbatim}

   This command emits color-coded help text showing the options, a brief
   description of each, and the default value of eac option.
   The text is extracted from the setup structures:

   \begin{verbatim}
      cfg::options::container s_test_options} 
      inisection::specification small_misc_data
      inisection::specification small_interaction_data
      inisection::specification small_comments = stock_comment_data();
      inisection::specification small_cfg66_data
      inisections::specification small_data
   \end{verbatim}

   The last specification collects all of the others and adds directory and
   descriptions.  Also included are all the specifications in the
   file \texttt{tests/rc\_spec.hpp}, which we will not show here.

   The first part of the help text shows the "global" options, which consist
   of internal stock options plus some additional test options:
   "list", "read", "write", and "tests.
   The global options are not associated with either an INI file
   or an INI section.

   Following the global options are specific options associated with an
   INI file and and INI section.  Here's a partial example:

   \begin{verbatim}
      rc:[misc] A number of miscellaneous options
       --manual-ports           Show real system ports, not 'usr' port names.
                                [false]
       --port-naming=v          Port amount-of-label showing. [short]
   \end{verbatim}

   The "rc" represents the INI file, which can have any name, but with an
   extension of \texttt{.rc}.
   For a given program, there can be only one INI file with that extension.
   The "[misc]" represents an INI section in that INI file.

\subsection{cfg::inisection}
\label{subsec:cfg_namespace_inisection}

   \texttt{cfg::inisection}  contains a \texttt{specification} structure that
   provides the name of a section, it's description, and
   a container of options.
   It is represent in a configuration file by a section or tag name
   enclosed in brackets:  \texttt{[output-ports]} as an example.
   Each section can also define a file extension
   (e.g. \texttt{.rc}) to indicate its locus.

\subsection{cfg::inisections}
\label{subsec:cfg_namespace_inisections}

   The \texttt{inisections.hpp} file has been split,
   and no long declares four classes.

   \texttt{cfg::inisections} defines four types:

   \begin{itemize}
      \item \texttt{specref}.
          \texttt{std::reference\_wrapper<inisection::specification>}.
      \item \texttt{specrefs}.
          \texttt{std::vector<specref>}.
      \item \texttt{sectionlist}.
         \texttt{std::vector<inisection>}.
      \item \texttt{specification}.
         This structure holds the configuration file's extension,
         directory, base name, description, and a vector of
         \texttt{specrefs}.
   \end{itemize}

   \texttt{cfg::inisections} holds a list of \texttt{inisection} objects.
   It represents all of the files, and all of
   the options that are held by the files.
   The options are in a single list,
   and the INI items look them up by name.

   There is a lot to this module.
   For now, see the comment in the \texttt{.hpp} and \texttt{.cpp} files.

\subsection{cfg::memento}
\label{subsec:cfg_namespace_memento}

   The \texttt{cfg::memento} template class is a small object that holds one
   copy of a state object.
   It provides these functions:

   \begin{itemize}
      \item \texttt{memento (const TYPE \& s)}.
      \item \texttt{bool set\_state (const TYPE \& s)}.
      \item \texttt{const TYPE \& get\_state () const}.
   \end{itemize}

   The \texttt{cfg::history} class stores a "history list":
   \texttt{std::deque<memento<TYPE>>}.

\subsection{cfg::options}
\label{subsec:cfg_namespace_options}

   The \texttt{cfg::options} holds a number of items needed for the
   specification, reading, and writing of options.
   The options can be read from configuration file or from the command-line.
   These items are nested in the class:

   \begin{itemize}
      \item \textsl{Static data}.
         Flags and numbers are provided to indicate if the option is enabled
         and how it is to be output in a nice format into an INI file.
      \item \textsl{kind}.
         Indicates if the option is boolean, numerical, a filename, list,
         string and some others. This enumeration makes it easier to
         process the option.
      \item \textsl{spec}.
         This nested structure contains these values. For brevity, the
         \texttt{option\_} portion of the name and the type are not shown:
         \begin{itemize}
            \item \texttt{code}.
               Optional single-character name.
            \item \texttt{kind}.
               Is it boolean, integer, string...?
            \item \texttt{cli\_enabled}.
               Normally true; false disables.
            \item \texttt{default}.
               Either a value or "true"/"false".
            \item \texttt{value}.
               The actual value as parsed.
            \item \texttt{read\_from\_cli}.
               Option already set from CLI.
            \item \texttt{modified}.
               Option changed since read/save.
            \item \texttt{desc}.
               A one-line description of option.
            \item \texttt{built\_in}.
               This option is present in all apps.
         \end{itemize}
      \item \textsl{option}.
         The \texttt{cfg::options::option} is a simple pair of the
         name of the option and the \texttt{spec} that describes it.
      \item \textsl{container}.
         The \texttt{cfg::options::container} is a map (dictionary) of
         option \texttt{spec}s keyed bythe name of the option.
   \end{itemize}

   Also specified are the name of the source file and the name of the
   source section in that file.

   Included are quite a number of functions for looking up option values
   and option characteristics.
   Also include are free functions to make options.

   The \texttt{options.cpp} module not only contains many comments explaining
   the module, but also a statically-initialized list of
   default options that any application can use.
   It is also a great example of how to creat a list of options.

\subsection{cfg::palette}
\label{subsec:cfg_namespace_}

   The \texttt{cfg::palette} template class can be used to define
   a palette of color code pair with a platform-specific color class
   such as \texttt{QColor} from \textsl{Qt}.
   This is a feature copped from \textsl{Seq66}.

\subsection{cfg::recent}
\label{subsec:cfg_namespace_recent}

   \texttt{recent} provides for the managment of a list of recently-used
   items.

   TODO.

%-------------------------------------------------------------------------------
% vim: ts=3 sw=3 et ft=tex
%-------------------------------------------------------------------------------

%-------------------------------------------------------------------------------
% cli
%-------------------------------------------------------------------------------
%
% \file        cli.tex
% \library     Documents
% \author      Chris Ahlstrom
% \date        2024-04-16
% \update      2024-07-20
% \version     $Revision$
% \license     $XPC_GPL_LICENSE$
%
%     Provides a description of the entities in the cfg66 library.
%
%-------------------------------------------------------------------------------

\section{Cli Namespace}
\label{sec:cli_namespace}

   This section provides a useful walkthrough of the \texttt{cli} namespace of
   the \textsl{cfg66} library.
   In addition, a \texttt{C}-only module is provided.

   Here are the classes (or modules) in this namespace:

   \begin{itemize}
      \item \texttt{cliparser\_c}
      \item \texttt{cli::parser}
      \item \texttt{cli::multiparser}
   \end{itemize}

\subsection{cliparser\_c Module}
\label{subsec:cli_namespace_c}

   This module is actually a \texttt{C++} module that implements a number of
   \texttt{extern "C"} functions.
   The functions themselves access an internal and hidden
   \texttt{cli::parser} object and call its member functions to
   perform the functions.

   This module provides a \texttt{C} structure that mirrors
   \texttt{options::spec}, plus some free functions to access this
   structure.

\subsection{cli::parser}
\label{subsec:cli_namespace_parser}

   The \texttt{cli::parser} class contains a number of options in a
   \texttt{cfg::options} object.
   Many of the member functions are pass-alongs to this object.
   This parser is meant for simple usage; it has no support for accessing
   configuration files, and supports just one set of options.

\subsection{cli::parser / Global Options}
\label{subsec:cli_namespace_parser_global_options}

   The \texttt{cli::parser} also provides support for options that can
   be command to all applications:

   \begin{itemize}
      \item \texttt{--version}.
         Provides the version of the application.
      \item \texttt{--help}.
         Emits help text. This help text is assembled from
         the specification structures set up according to the
         \texttt{cfg::options::container}.
      \item \texttt{--description}.
         This option is meant to show the description fields of the
         options.
      \item \texttt{--verbose}.
         Indicates to emit additional information about the run of the
         application.
      \item \texttt{--inspect}.
         This option is meant to activate debugging code.
         The applications determines what this means.
      \item \texttt{--investigate}.
         This option is meant to activate temporary debugging code for
         the current purpose desired by the programmer.
         The applications determines what this means.
      \item \texttt{--log}.
         This option enables a log file for recording error and information
         output.
   \end{itemize}

   The \texttt{parse()} function looks for stock option such as
   \texttt{--help} and \texttt{--option log filename}.
   The \texttt{--} sequence terminates a list of options.

   It is meant to be similar to \texttt{getopt(3)}, but much more flexible and
   perhaps easier to set up.

   The caller can call the following member functions:

   \begin{itemize}
      \item \texttt{show\_information\_only()}.
         This function returns a boolean value that is true if
         help, description, or version were requested.
         Generally, if this information is shown, the application does
         nothing but show the desired information, and exits.
      \item \texttt{version\_request()}.
         Indicates if version information was requested.
      \item \texttt{help\_request()}.
         Indicates if application help was requested.
         The command-line help text is shown.
         The caller can also provide further help if desired.
      \item \texttt{description\_request()}.
         Indicates if description information was requested.
      \item \texttt{verbose\_request()} and \texttt{verbose()}.
         Indicates if extra output (either command line or via
         dialog boxes) was requested.
      \item \texttt{inspect\_request()}.
         Indicates if inspection was specified was requested.
         Generally can be used to output data values as appropriate.
      \item \texttt{investigate\_request()}.
         Indicates if investigation was specified was requested.
         Defined by the application.
      \item \texttt{use\_log\_file()} and \texttt{log\_file()}.
         Indicates that standard I/O should be redirected to a log file.
   \end{itemize}

   There are functions to add, verify, and set option values:

   \begin{itemize}
      \item \texttt{add()}.
      \item \texttt{verify()}.
      \item \texttt{set\_value()}.
      \item \texttt{change\_value()}.
      \item \texttt{clear()}.
      \item \texttt{unmodify()}.
      \item \texttt{unmodify\_all()}.
   \end{itemize}

   There are also functions to test and retrieve option values:

   \begin{itemize}
      \item \textsl{string}.
         All options are encoded as strings.
         Strings can be retrieved using \texttt{cli::parser::value()}.
         The default string can be retrieved by \texttt{default\_value()}.
      \item \textsl{boolean}.
         \texttt{is\_boolean()} tests for a value being boolean.
      \item \textsl{other, to do}.
         There are many more value-related functions in the
         \texttt{cfg::options} class, but they use a reference to
         a \texttt{cfg::options::spec} structure, and can be accessed
         only by getting the option and spec reference.
         See \sectionref{subsec:cfg_namespace_options}.
   \end{itemize}

   The format of the default value is either a simple string showing the
   default value, or a \textsl{range} string such as "0<=10<=99", where
   the middle value is the default value.

\subsection{cli::parser / Usage}
\label{subsec:cli_namespace_parser_usage}

   This section provides a walk through the test application
   \texttt{cliparser\_test}. It starts with an \texttt{appinfo} function
   to see the name that will appear in console messages:

   \begin{verbatim}
      cfg::set_client_name("cli");
      cfg::set_app_version("0.2.0");
   \end{verbatim}

   Next, a parser is created via:

   \begin{verbatim}
      cli::parser clip(s_test_options);
   \end{verbatim}

   The \texttt{s\_test\_options} list is defined in the \texttt{test\_spec}
   header file. Here is an excerpt. Note that we could use an anonymous
   namespace instead of the keyword \texttt{static}.

   \begin{verbatim}
      static cfg::options::container s_test_options
      {
          {
              {
                  "alertable",
                  {
                      'a', cfg::options::kind::boolean, cfg::options::enabled,
                      "false", "", false, false,
                      "If specified, the application is alertable.", false
                  }
              },
              {
                  "canned-code",
                  {
                      'c', cfg::options::kind::boolean, cfg::options::enabled,
                      "true", "", false, false,
                      "If specified, the application employs canned code.", false
                  }
              },
              . . .
          }
      };
   \end{verbatim}

   As an example, this list provides the \texttt{-a} and \texttt{--alertable}
   option, which is a boolean option and is enabled. It has a default value
   of "false", which is set once the list is initialized.
   The values for "set-from-cli" and "dirty" are false by default at first.
   After the one-line description, the boolean indicates if the options
   is a built-in (global, stock) option. 
   That value is set only in the \texttt{global\_options()} function
   in the \texttt{options} module.
   The options in this list are added to the global options.

   After construction, the parser can be applied to the command-line
   arguments:

   \begin{verbatim}
       bool success = clip.parse(argc, argv);
   \end{verbatim}

   This can modify (set the value, raise the "set-from-cli" and "dirty"
   flags) options based on the command-line.

   Not all options will have a code, in general, especially if there are a
   large number of options. Duplicates are checked for.
   The existing codes can be listed:

   \begin{verbatim}
      std::string msg = "Option codes: " + clip.code_list();
   \end{verbatim}

   The command line can also be scanned to see if an option is present.
   The last parameter indicates if the option is required to exist:

   \begin{verbatim}
      bool findme_active = clip.check_option(argc, argv, "find-me", false);
   \end{verbatim}

   Built-in (global) options that make the application show something and
   then quit can be tested:

   \begin{verbatim}
      if (clip.show_information_only())
      {
          if (clip.help_request()) . . .
          if (clip.description_request()) . . .
          if (clip.version_request()) . . .
      }
   \end{verbatim}

   Some other built-in options:

   \begin{verbatim}
      if (clip.verbose_request()) . . .
      if (clip.inspect_request()) . . .
      if (clip.investigate_request()) . . .
      if (clip.use_log_file()) . . .
   \end{verbatim}

   Debug text can be shown; it lists all the options, their values, their
   default values, and if the value has been modified:

   \begin{verbatim}
      std::string dbgtxt = clip.debug_text(cfg::options::stock);
   \end{verbatim}

   Values can also be set or changed by the application itself, rather
   than from the command-line. It can be retrieved by the long option
   name or the option code.

   \begin{verbatim}
      success = clip.change_value("username", "C. Ahlstrom");
      std::string name = clip.value("u");    // or "username"
      success = name == "C. Ahlstrom";
   \end{verbatim}

   If an error occurs, which can be checked by a function's return code
   or by a call to \texttt{cli::parser::error\_msg()}, then the message can
   be retrieved for display:

   \begin{verbatim}
        std::string errmsg = clip.error_msg();
        util::error_message(errmsg);
   \end{verbatim}

\subsection{cli::multiparser}
\label{subsec:cli_namespace_multiparser}

   The \texttt{cli::multiparser} class extends \texttt{cli::parser} in
   a number of ways, in order to support a suite of command-line options
   covering mutltiple configuration files.

   Support is provided to look up the long option name from an option code
   character. (The \texttt{cfg::options} class also provides this, but it
   is not directly exposed to \texttt{cli::parser}.

   \begin{verbatim}
      using codes = std::map<char, std::string>;
      codes & code_mappings();
   \end{verbatim}

   This function is used in the \texttt{multiparser} override of the
   the virtual \texttt{cli::parser::parse()} function.

   In order to support multiple configuration files and multiple
   configuration sections, we need to way to find out in which file and section
   an option resides.

   \begin{verbatim}
      using duo = struct
      {
         std::string config_type;
         std::string config_section;
      };
      using names = std::map<std::string, duo>;
      names & cli_mappings();
   \end{verbatim}

   This function is used to find the desired option set, a pointer to
   the \texttt{cfg::options} for a particular file and section.

   For a walk-through, see the section concerning the "INI set test".
   TO DO.


%-------------------------------------------------------------------------------
% vim: ts=3 sw=3 et ft=tex
%-------------------------------------------------------------------------------

%-------------------------------------------------------------------------------
% session
%-------------------------------------------------------------------------------
%
% \file        session.tex
% \library     Documents
% \author      Chris Ahlstrom
% \date        2024-04-16
% \update      2024-08-25
% \version     $Revision$
% \license     $XPC_GPL_LICENSE$
%
%     Provides a description of the entities in the cfg66 library.
%
%-------------------------------------------------------------------------------

\section{Sessions}
\label{sec:sessions}

\subsection{Session Naming Conventions}
\label{subsec:session_naming_conventions}

   The \textsl{Cfg66} session support relies heavily on directories.
   Here are the shorthand names we will use for them; "\$username" is the name
   of the user account; "\$appname" is the short name of the application;
   "\$clientname" is "\$appname" with a distinguishing wart (e.g \texttt{-123};
   "\$app-version" is "\$appname" with the version appended
   (e.g. \texttt{-0.99}.

   \begin{itemize}
      \item \texttt{\$prefix}.
         The base installation directory of the application that uses the
         \textsl{Cfg66} library.
         \begin{itemize}
            \item \textsl{Linux}:
               \texttt{/usr/} or \texttt{/usr/local}.
               Some common subdirectories:
               \begin{itemize}
                  \item \texttt{bin/}. 
                     Contains the application's executable file.
                  \item \texttt{include/\$app-version/}.
                     Contains the directory hierarchy of the application,
                     to support building the application from source code.
                  \item \texttt{lib/\$app-version/}.
                     Contains the libraries needed for the application.
                  \item \texttt{man/}. 
                     Contains the application's man pages.
                  \item \texttt{doc/\$app-version/}. 
                     Contains the application's documentation, which could
                     include HTML text, PDF files, tutorial files, and more.
                  \item \texttt{share/\$app-version/}. 
                     Contains stock data for the application, including
                     icons, pixmaps, sample data files, and sample
                     configuration files.
               \end{itemize}
            \item \textsl{BSD}:
               \texttt{/usr/local}.
                  \textsl{FreeBSD} reserves \texttt{/usr/} for system
                  files. Otherwise, the layout is about the same as for
                  \textsl{Linux}.
            \item \textsl{Windows}:
               The layout for \textsl{Windows} is notably different.
               \texttt{C:/Program Files/}. 
               The user can choose a disk drive other than \texttt{C:}.
               We currently aim to support only 64-bit applications.
               Here are the common directories as set up by \textsl{Cfg66}.
               \begin{itemize}
                  \item \texttt{C:/Program Files/\$appname/}. 
                     We might consider installing the executable into a
                     versioned subdirectory of this one.
                  \item \texttt{C:/Program Files/\$appname/data}. 
                     Contains all the installed items that are in
                     \texttt{share/} in UNIXen.
               \end{itemize}
         \end{itemize}
      \item \texttt{\textasciitilde}.
         The HOME directory of the user as supported in various operating
         systems.
         An equivalent symbol is \texttt{\$HOME}.
         \begin{itemize}
            \item \textsl{Linux}:
               \texttt{/home/username/}
            \item \textsl{BSD}:
               \texttt{/home/username/}
            \item \textsl{Windows}:
               \texttt{C:/Users/username/}
         \end{itemize}
      \item \texttt{\$home}.
         The directory of the configuration files as supported by default in
         \textsl{Cfg66}.
         It can be modified by a command-line option as described in the
         next section.
         \begin{itemize}
            \item \textsl{Linux}:
               \texttt{/home/username/.config/\$appname}.
            \item \textsl{BSD}:
               \texttt{/home/username/.config/\$appname}.
            \item \textsl{Windows}:
               \texttt{C:/Users/\$username/AppData/Local/\$appname}.
         \end{itemize}
      \item \texttt{\$directory}.
         This subdirectory can be specified in the \texttt{.session} file.
         Note that session files are \textsl{always} stored in
         \textsl{\$home}, even if \texttt{\$home} is modified by a
         command-line option.
         If \texttt{\$directory} is not empty,
         it replaces \texttt{\$home} as the home-directory of
         the application while it is running.
         It can have a full path if desired.
         If it has no path, then it is a subdirectory of \texttt{\$home}.
   \end{itemize}

   Confused? I sure am!

\subsection{Session Description}
\label{subsec:session_description}

   A session provides a specific configuration for an application.
   The environment is contained and specified in a
   \textsl{session directory}, also known as a
   \textsl{home directory}.
   As an example, home directories for \textsl{Seq66} running on its
   own or inside an \textsl{NSM} (\cite{nsm}) session:

   \begin{verbatim}
      /home/user/.config/seq66
      /home/user/NSM Sessions/Seq66/seq66.nGJDW
   \end{verbatim}

   Note that the \texttt{seq66.nGJDW} directory is created by \texttt{NSM}.
   \textsl{Cfg66} does not create a configuration directory by default, 
   but it has some options:

   \begin{itemize}
      \item \texttt{--home \textless path \textgreater}.
         This option changes the "home" directory if the argument includes 
         a path, or appends a directory name to the default "home" if it
         does not include a path.
         This changes the directory where the \texttt{.session} files are
         found.
         If \texttt{NSM} is running (as shown above), that directory
         becomes \texttt{\$home}.
      \item \texttt{--session \textless name[.session] \textgreater}.
         This option selects a session file from \texttt{\$home}.
         The default session file \texttt{\$appname.session}.
         This option is \textsl{not} to be used if \texttt{NSM} is running.
   \end{itemize}

   The "home" directory could have subdirectories, such as
   \texttt{config}, \texttt{audio}, \texttt{midi}, \texttt{data}, etc.
   These are application specific, and can be listed in the 
   \texttt{\$appname.session} file.

   \textsl{Cfg66} supports a \texttt{.session} file that lists
   the names, locations, and activity of the various other INI files.
   With that setup, we would need a session file for each session.
   With \texttt{NSM} the session setup is determined by
   \texttt{NSM}, and the session file is always \texttt{\$appname.session}.
   Use cases to support:

   \begin{itemize}
      \item \textbf{Standalone application}.
         The application automatically refers to a set of configuration files
         in a "home" directory,
         either \texttt{~/home/user/.config/seq66} or
         either \texttt{~/home/user/.config/seq66/config}.
      \item \textbf{Session selection}.
         All session information resides in a known file in the home
         directory, \texttt{\$appname.session}, but a selection of the session
         file can be made at startup.
      \item \textbf{NSM session}.
         There is a session manager that communicates the session directory
         to the application.
   \end{itemize}

\section{Session Namespace}
\label{sec:session_namespace}

   What is a session?
   It is an environment in which an application runs, and is one of many
   possible environments for the application.
   \textsl{Cfg66} provides rudimentary management of the characteristics of
   an application. Compare that to a true session manager such as the
   \textsl{New Session Manager}(\cite{nsm}, NSM), which coordinates not only
   the main application, but a group of applications and their configuration
   and connections, recreating recreate complex setups based on the
   contents of a particular session directory..
   Here, we want a way to provide various setups for an application.
   Of course, these setups can be placed under the control of a more complex
   session manager.

   This section provides a useful walkthrough of the \texttt{session} namespace of
   the \textsl{cfg66} library.
   In addition, a \texttt{C}-only module is provided.

   Here are the classes (or modules) in this namespace:

   \begin{itemize}
      \item \texttt{session::climanager}
      \item \texttt{session::configuration}
      \item \texttt{session::directories}
      \item \texttt{session::manager}
   \end{itemize}

\subsection{session::climanager}
\label{subsec:session_namespace_climanager}

   The \texttt{session::climanager} class is derived from
   \texttt{session::manager} and overrides a number of virtual functions
   It also provides a function to read a configuration file.
   It provides a \texttt{run()} loop which does nothing
   but check for calls to close and save the session and wait for a small
   polling period.

\subsection{session::configuration}
\label{subsec:session_namespace_configuration}

   The \texttt{session::configuration} class is derived from
   \texttt{cfg::basesettings}.
   It contains a \texttt{session::directories} management class
   and a set of
   \texttt{directories::entries} items.
   Options for help, a log-file, and auto-save are provided.

\subsection{session::directories}
\label{subsec:session_namespace_directories}

   The \texttt{session::directories} class manages a set of
   \texttt{entry} directory item.

   Each \texttt{entry} specifies:

   \begin{itemize}
      \item Name of the section covered by a configuration file.
      \item It active/inactive status.
      \item The directory for the files(s).
      \item The base name of the file(s).
      \item The optional extension of the file(s).
   \end{itemize}

   Provides the full path specification of each file, constructed
   from the entries, and keyed by the section name.
   The "home" configuration path and the session path are also specified.

   The \texttt{directories.cpp} module explains the directory layouts
   and provides default "rc" and "log" directory specifications.

\subsection{session::manager}
\label{subsec:session_namespace_manager}

   The \texttt{session::manager} class is base class for providing an
   application with "session" information, where a session is a group of
   configuration items that allow an application to run in a sequestered
   environment. Think of the \textsl{JACK Session Manager} or the 
   \textsl{New/Non Session Manager}.

   The base session manager class holds the following information:

   \begin{itemize}
      \item \texttt{session::configuration}.
         See the section about this class above.
      \item \texttt{cli::parser}.
         Provides access to the command-line parser.
      \item \textsl{Capabilities}.
         This application-dependent string publishes some information about
         the application. Useful with the
         \textsl{New/Non Session Manager}.
      \item \textsl{Manager name}.
         The name of the session manager. For example, it is returned by the
         \textsl{New/Non Session Manager}.
      \item \textsl{Manager path}.
         This item holds the directory where the session information is to
         be stored.
         For example, the \textsl{New/Non Session Manager} returns a string
         like
         \texttt{/home/user/NSM Sessions/JackSession/seq66.nUKIE}.
      \item \textsl{Display name}.
         This is the name of the session to be displayed, such as
         \textsl{JackSession} in the string above.
      \item \textsl{Client ID}.
         This is the name of the client (e.g. for managing port
         connections), such as \textsl{Seq66} or \textsl{seq66.nUKIE}.
   \end{itemize}

   Also included are indicators for \texttt{--help}, dirty status, and error
   messages.

   A large number of \texttt{virtual} members functions are included.
   Some of the important functions are for the following actions:

   \begin{itemize}
      \item Parsing an option (configuration) file.
      \item Parsing the command line.
      \item Creating, writing, and reading a configuration file.
      \item Creating, saving, and closing a session.
      \item Creating a session directory.
      \item Creating a "project".
      \item Creating a manager.
      \item Running a session, often in a loop or a GUI thread.
   \end{itemize}
   
   The \texttt{manager.cpp} module contains a short
   statically-initialized list of
   default options.

   Note that there are currently a number of "To Dos" in this class.

\section{Session Walkthroughs}
\label{sec:session_walkthroughs}

	While the session layouts that can be supported are many, we will
	walk through only the most common (i.e. tested) scenarios.

   The first thing to do when the application starts is determine
   the \texttt{\$home} directory.

   The first step is to provide a \texttt{cfg::appinfo} structure
   that defines at least the app-name.
   If desired, one can also provide non-empty string values for
   the app-version, config-section name, home directory and home
   configuration file, as well as other items. If empty, defaults
   are used or assembled.

   The next step (optional) is to call \texttt{user\_home(appfolder)}
   to change the folder name.

   TODO.
   TODO.
   TODO.
   TODO.

%-------------------------------------------------------------------------------
% vim: ts=3 sw=3 et ft=tex
%-------------------------------------------------------------------------------

%-------------------------------------------------------------------------------
% util
%-------------------------------------------------------------------------------
%
% \file        util.tex
% \library     Documents
% \author      Chris Ahlstrom
% \date        2024-04-16
% \update      2024-05-16
% \version     $Revision$
% \license     $XPC_GPL_LICENSE$
%
%     Provides a description of the entities in the cfg66 library.
%
%-------------------------------------------------------------------------------

\section{Util Namespace}
\label{sec:util_namespace}

   This section provides a useful walkthrough of the \texttt{util} namespace of
   the \textsl{cfg66} library.

   Here are the classes (or modules) in this namespace:

   \begin{itemize}
      \item \texttt{util::bytevector class}
      \item \texttt{util::filefunctions module}
      \item \texttt{util::msgfunctions module}
      \item \texttt{util::named\_bools class}
      \item \texttt{util::strfunctions module}
   \end{itemize}

   All of the modules are \texttt{C++} modules with free functions
   in the \texttt{util} namespace.

\subsection{util::bytevector}
\label{subsec:util_namespace_bytevector}

   The \texttt{util::bytevector} class
   provides an \texttt{std::vector} of "bytes" (\texttt{unsigned char}) with
   functions to put bytes into the vector and read them out.
   There are also functions to read a file and write the vector to the
   file.

   The bytes are treated as a stream of big-endian data.
   Integers are extracted from the bytes a byte at a time, starting with
   the most significant byte.
   Since this data is big-endian, it is suitable for use with MIDI
   files and network data streams.

\subsection{util::filefunctions module}
\label{subsec:util_namespace_filefunctions}

   The \texttt{util::filefunctions} module contains a large number of
   function dealing file-names and files.

   The file functions are basically wrappers around the \texttt{C FILE *}
   API.

   The file-name functions are useful for building paths and for splitting
   paths into parts.

   Really, just skim the \texttt{filefunctions} modules to learn what is
   there.  They include every convenient function we needed in implementing
   \textsl{Seq66}.

\subsection{util::msgfunctions module}
\label{subsec:util_namespace_msgfunctions}

   The \texttt{util::msgfunctions} module defines functions for writing
   messages to the console along with tags showing the short name of the
   application that wrote them, and in color.

   Also included are some "async safe" functions for output and for
   converting unsigned numbers to string arrays.

\subsection{util::named\_bools}
\label{subsec:util_namespace_named_bools}

   The \texttt{util::name\_bools} class
   makes it easy to look up and set a "small" number of
   boolean values by name.

   This class could be useful if one does not want the full capability
   of the classes in the \texttt{cfg} namespace.

\subsection{util::strfunctions module}
\label{subsec:util_namespace_strfunctions}

   The \texttt{util::strfunctions} module defines functions for manipulating
   strings: tokenization, left/right space trimming, conversion between
   strings and values with the added feature of defaulting, word-wrapping,
   and formatting of \texttt{std::string} values without using
   \texttt{std::stringstream}.

%-------------------------------------------------------------------------------
% vim: ts=3 sw=3 et ft=tex
%-------------------------------------------------------------------------------

%-------------------------------------------------------------------------------
% tests
%-------------------------------------------------------------------------------
%
% \file        tests.tex
% \library     Documents
% \author      Chris Ahlstrom
% \date        2024-04-16
% \update      2024-04-16
% \version     $Revision$
% \license     $XPC_GPL_LICENSE$
%
%     Provides a pointed description of some tests, which also helps understand
%     usage of the cfg66 library.
%
%-------------------------------------------------------------------------------

\section{Cfg66 Tests}
\label{sec:cfg66_tests}

   This section provides a useful walkthrough of the testing
   of the \textsl{cfg66} library.
   They illustrate the various ways in which the \textsl{Cfg66} library
   can be used by a developer.

   The tests so far are these executables:

    \begin{itemize}
        \item \texttt{cliparser\_test\_c}
        \item \texttt{cliparser\_test}
        \item \texttt{history\_test}
        \item \texttt{ini\_test}
        \item \texttt{manager\_test}
        \item \texttt{options\_test}
    \end{itemize}

    These tests are supported by data structures define in the following
    header files:

    \begin{itemize}
        \item texttt{ctrl\_spec.hpp}
        \item texttt{drums\_spec.hpp}
        \item texttt{mutes\_spec.hpp}
        \item texttt{palette\_spec.hpp}
        \item texttt{playlist\_spec.hpp}
        \item texttt{rc\_spec.hpp}
        \item texttt{session\_spec.hpp}
        \item texttt{test\_spec.hpp}
        \item texttt{usr\_spec.hpp}
    \end{itemize}

    These header files will be discussed as needed in the following sections.

\subsection{Cfg66 cliparser\_test\_c Test}
\label{subsec:cfg66_tests_cliparser_test_c}

    This test of the \texttt{C} command-line interface uses the free
    functions in \texttt{cliparser\_c.h}. The test module itself
    sets up a small set of test options:

    \begin{verbatim}
    static options_spec s_test_options [] =
    {
        //   Name          Code  Kind     Enabled   Default     Value      Dirty
        {
            "alertable",    'a', "boolean", true,   "false",    "false",   false,
            "If specified, the application is alertable."
        }, . . .
    };
    \end{verbatim}

    The test makes changes to the options and verifies that they
    took hold.
    The test command is simple:

    \begin{verbatim}
        $ ./build/test/cliparser_test_c
    \end{verbatim}

    It shows the changes and a result statement.

\subsection{Cfg66 cliparser\_test Test}
\label{subsec:cfg66_tests_cliparser_test}

    This test uses the following tests options (only one is shown)
    static initialization:

    \begin{verbatim}
    static cfg::options::container s_test_options
    {
        //   Name, Code,  Kind, Enabled,
        //   Default, Value, FromCli, Dirty,
        //   Description, Built-in
        {
            {
                "alertable",
                {
                    'a', "boolean", cfg::options::enabled,
                    "false", "false", false, false,
                    "If specified, the application is alertable.",
                    false
                }
            }, . . .
        }
    };
    \end{verbatim}

    It serves as a good example of how to create a list of options.
    More flexible than GNU's getopt setup and simplifies generating help
    text.

    The test makes changes to the options and verifies that they
    took hold.
    The test command is not as simple as the \texttt{C} version, as
    verbosity is needed to see the changes:

    \begin{verbatim}
        $ ./build/test/cliparser_test --verbose
    \end{verbatim}

    It shows the changes and a result statement.
    This test needs a little bit of cleanup.

\subsection{Cfg66 history\_test Test}
\label{subsec:cfg66_tests_history_test}

    The history test also sets up a test options "array".
    Then it makes changes to the options, such as changing variables,
    undoing the change, and redoing the change.
    It shows the changes and a result statement.

    See this test file for some "To do" items.

\subsection{Cfg66 ini\_test Test}
\label{subsec:cfg66_tests_ini_test}

   This test program uses all of these "data" headers:

   \begin{itemize}
      \item texttt{ctrl\_spec.hpp}
      \item texttt{drums\_spec.hpp}
      \item texttt{mutes\_spec.hpp}
      \item texttt{palette\_spec.hpp}
      \item texttt{playlist\_spec.hpp}
      \item texttt{rc\_spec.hpp}
      \item texttt{session\_spec.hpp}
      \item texttt{usr\_spec.hpp}
    \end{itemize}

    It defines these static test items:

    \begin{itemize}
      \item \texttt{cfg::options::container s\_test\_options}.
         This sets up a single option called "test",
         used as a command-line option.
      \item \texttt{cfg::inisection::specification s\_simple\_ini\_spec}.
         This sets up an \texttt{[experiments]} section with a number of
         option-variable definitions.
      \item \texttt{cfg::inisection::specification s\_section\_spec}.
         This sets up an \texttt{[section-test]} section.
      \item \texttt{cfg::inifile::specification exp\_file\_data}
         This items sets up the "experiment" configuration directory
         using
         \texttt{cfg::inisection::specification s\_simple\_ini\_spec}.
         \texttt{cfg::inisection::specification s\_section\_spec}.
    \end{itemize}

    Additional sections are defined and add to a
    \texttt{cfg::inifile::specification} declaration.

    Hmmm. some are unsued.

\subsection{Cfg66 manager\_test Test}
\label{subsec:cfg66_tests_manager_test}

   This test defines 
   \texttt{cfg::appinfo s\_application\_info}.
   This is used here:
   \texttt{cfg::initialize\_appinfo(s\_application\_info, argv[0])}.

   The "To do" here is to actually implement \texttt{simple\_smoke\_test}.

\subsection{Cfg66 options\_test Test}
\label{subsec:cfg66_tests_options_test}

   This test program uses only the "data" header
   \texttt{test\_spec.hpp} which defines
   \texttt{cfg::options::container s\_test\_options}.
   This container is used to initialize a
   \texttt{cli::parser}.
   That object then gets the command-line arguments.

   Obviously, we still have a lot of work to do with these tests.

%-------------------------------------------------------------------------------
% vim: ts=3 sw=3 et ft=tex
%-------------------------------------------------------------------------------


\section{Summary}
\label{sec:summary}

   Contact: If you have ideas about \textsl{Cfg66} or a bug report,
   please email us (at \url{mailto:ahlstromcj@gmail.com}).

% References

%-------------------------------------------------------------------------------
% references
%-------------------------------------------------------------------------------
%
% \file        references.tex
% \library     Documents
% \author      Chris Ahlstrom
% \date        2024-04-15
% \update      2024-07-24
% \version     $Revision$
% \license     $XPC_GPL_LICENSE$
%
%     Provides the References section of the Cfg66 manual. Rather
%     than use the bibtex package, our small set of references uses a
%     simpler method.
%
%-------------------------------------------------------------------------------

\section{References}
\label{sec:references}

   The \textsl{Cfg66} reference list.

{\RaggedRight
\begin{thebibliography}{99}

   \bibitem{deque}
   Twinkle Sharma.
   \emph{Deque in C++}
   \url{https://www.scaler.com/topics/cpp/deque-in-cpp/}.
   2024.

   \bibitem{dpatterns}
   Erich Gamma, Richard Helm, Ralph Johnson, and John Vlissides.
   \emph{Design Patterns: Elements of Reusable Object-Oriented Software.}
   \url{Use your search engine}.
   Start with page 62.
   1994.

%  \bibitem{memento}
%  To do.
%  \emph{The Memento Design Pattern}
%  \url{https://}.
%  2023.

   \bibitem{seq66}
   Chris Ahlstrom.
   \emph{A reboot of the Seq24 project as "Seq66".}
   \url{https://github.com/ahlstromcj/seq66/}.
   2015-2024.

\end{thebibliography}
}

%-------------------------------------------------------------------------------
% vim: ts=3 sw=3 et ft=tex
%-------------------------------------------------------------------------------


\printindex

\end{document}

%-------------------------------------------------------------------------------
% vim: ts=3 sw=3 et ft=tex
%-------------------------------------------------------------------------------
